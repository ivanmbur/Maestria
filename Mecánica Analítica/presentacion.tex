\documentclass{beamer}
\usetheme{Boadilla}

\usepackage[utf8]{inputenc}
\usepackage{tikz-cd}

\title{Geometric Mechanics}
\author[I. M. Burbano]{Iván Mauricio Burbano Aldana\thanks{\href{mailto:im.burbano10@uniandes.edu.co}{im.burbano10@uniandes.edu.co}}}
\institute[Uniandes]{Universidad de los Andes}

\AtBeginSection[]
{
  \begin{frame}
    \frametitle{Table of Contents}
    \tableofcontents[currentsection]
  \end{frame}
}

\begin{document}

\frame{\titlepage}

\section{Motivation}

\begin{frame}

\frametitle{General Spaces}

Usually, $\mathbb{R}^n$ is not an appropriate space to describe the configuration of a system.

\begin{itemize}

\item Particle in a box $Q=\Lambda\subseteq\mathbb{R}^3$.

\item Pendulum $Q=S^1:=\{(x,y)\in\mathbb{R}^2\}$.

\item N identical particles $Q=\mathbb{R}^{3N}/S_N$. 

\end{itemize}

\end{frame}

\begin{frame}

\frametitle{A Coordinate Independent Formulation}

Coordinates are artifacts we use to understand the world. However, physical processes should be unaffected by the way we choose to describe them. Physical laws should thus be coordinate independent. 

\vspace{1cm}

We can actually express them without coordinates!

\end{frame}

\section{Differential Geometry}

\begin{frame}

\frametitle{Topological Spaces}

\begin{definition}
A topology on a set $X$ is a collection of subsets $\mathcal{O}$ of $X$ such that:

\begin{itemize}

\item both $X,\varnothing\in\mathcal{O};$

\item for every family $\{U_i\in\mathcal{O}|i\in I\}$ we have 
\begin{equation}
\bigcup_{i\in I}U_i\in\mathcal{O}\qquad\text{(Closed under unions);}
\end{equation}

\item for every $U_1,\dots,U_n\in\mathcal{O}$ we have
\begin{equation}
\bigcap_{i=1}^nU_i\in\mathcal{O}\qquad\text{(Closed under finite intersections).}
\end{equation}

\end{itemize}

\end{definition}

\end{frame}

\begin{frame}

\frametitle{Topological Jargon}

\begin{itemize}

\item Elements of a topology $\mathcal{O}$ are said to be open.

\item An open set $U$ which contains a point $p\in X$ is called a neighborhood of $p$.

\item A space equipped with a topology is said to be a topological space.

\end{itemize}

\end{frame}

\begin{frame}

\frametitle{Standard Topology on $\mathbb{R}^n$}

A subset $U\subseteq\mathbb{R}^n$ is said to be open if for every $p\in U$ there exists a radius $r\in(0,\infty)$ such that the open ball of radius $r$ centered at $p$
\begin{equation}
B_r(p):=\{q\in\mathbb{R}^n|\|q-p\|<r\}
\end{equation}
is contained in $U$.

\end{frame}

\begin{frame}

\frametitle{Why topology? Convergence}

After years of debate, mathematicians finally settled on topology being the most general setting for the study of convergence.

\begin{definition}
A sequence $(x_n)$ in a topological space $X$ is said to converge to $x\in X$ if for every open set $U\ni x$ (from now on called a neighborhood of $x$) there exists an $N\in\mathbb{N}$ such that for all $n\in\mathbb{N}$ greater than $N$ we have $x_n\in U$.
\end{definition}

\end{frame}

\begin{frame}

\frametitle{Why topology? Continuity}

\begin{definition}
A function $f:X\rightarrow Y$ between topological spaces is said to be continuous if $f^{-1}(V)\subseteq X$ is open for all open $V\subseteq Y$.
\end{definition}

\begin{definition}
Two topological spaces $X$ and $Y$ are said to be homeomorphic if there is a continous bijection $f:X\rightarrow Y$ whose inverse is also homeomorphic. 
\end{definition}

\end{frame}

\begin{frame}

\frametitle{Subspace Topology}

\begin{definition}
Let $Y\subseteq X$ be a subset of a topological space $X$. A subset $U\subset Y$ is said to be open (relative to $Y$) if there exists an open $V\subseteq X$ such that $U=V\cap Y$.
\end{definition}

\end{frame}

\begin{frame}

\frametitle{Topological Manifolds}

\begin{definition}
A locally euclidean space $M$ of dimension $n$ is a topological space where every point $p\in M$ has a neighborhood $U\ni p$ which is homeomorphic to an open subset of $\mathbb{R}^n$. A topological manifold of dimension $n$ is a Hausdorff second countable locally euclidean spaces of dimension $n$. 
\end{definition}

\end{frame}

\begin{frame}

\frametitle{Manifold Jargon}

\begin{itemize}

\item A chart on $M$ is a pair $(U,x)$, where $U\subseteq X$ is open and $x:U\subseteq M\rightarrow x(U)\subseteq\mathbb{R}^n$ is an homeomorphism. 

\item An atlas on $M$ is a collection of charts which covers $M$. 

\end{itemize}

\end{frame}

\begin{frame}[fragile]

\frametitle{Manifold Philosophy}

Use charts to verify properties on $M$ which are defined on $\mathbb{R}^n$. Restrict your atlas such that all charts agree on the properties.

\begin{tikzcd}
M \ar[r, "f"] \ar[d]\ar[u] & \mathbb{R} \\ 
\end{tikzcd}

\end{frame}

\end{document}
\documentclass{article}

\usepackage[utf8]{inputenc}
\usepackage[spanish]{babel}
\usepackage{slashed}
\usepackage{physics}

\author{Iván Mauricio Burbano Aldana}
\title{Mecánica Cuántica Avanzada\\Tarea 5}

\begin{document}

\maketitle

\begin{enumerate}

\item[4.8] Multiplicando por izquierda $\overline{u}_r(\vb{p'})\gamma^\mu$ a la ecuación (4.46) de \cite{Lahiri2005} se obtiene
\begin{equation}
0=\overline{u}_r(\vb{p'})\gamma^\mu(\slashed{p}-m)u_s(\vb{p})=\overline{u}_r(\vb{p'})\gamma^\mu\gamma^\nu p_\nu u_s(\vb{p})-\overline{u}_r(\vb{p'})\gamma^\mu m u_s(\vb{p}).
\end{equation}
Multiplicando por la derecha $\gamma^\mu u_s(\vb{p})$ a la ecuación (4.48) de \cite{Lahiri2005} se obtiene
\begin{equation}
0=\overline{u}_r(\vb{p'})(\slashed{p}'-m)\gamma^\mu u_s(\vb{p})=\overline{u}_r(\vb{p'})\gamma^\nu\gamma^\mu p'_\nu u_s(\vb{p})-\overline{u}_r(\vb{p'})\gamma^\mu m u_s(\vb{p}).
\end{equation}
Sumandolas se concluye que
\begin{align}\label{ec:u_original}
\begin{split}
0=&\overline{u}_r(\vb{p'})\gamma^\mu\gamma^\nu p_\nu u_s(\vb{p})-\overline{u}_r(\vb{p'})\gamma^\mu m u_s(\vb{p})\\
+&\overline{u}_r(\vb{p'})\gamma^\nu\gamma^\mu p'_\nu u_s(\vb{p})-\overline{u}_r(\vb{p'})\gamma^\mu m u_s(\vb{p})\\
=&\overline{u}_r(\vb{p'})\qty(\gamma^\mu\gamma^\nu p_\nu+\gamma^\nu\gamma^\mu p'_\nu) u_s(\vb{p})-2\overline{u}_r(\vb{p'})\gamma^\mu m u_s(\vb{p}).
\end{split}
\end{align}
En particular, si $\vb{p}=\vb{p'}$
\begin{equation}
0=\overline{u}_r(\vb{p})2g^{\mu\nu} p_\nu u_s(\vb{p})-2m\overline{u}_r(\vb{p})\gamma^\mu u_s(\vb{p}).
\end{equation}
Dividiendo por 2 y subiendo el índice del momento entonces es claro que
\begin{equation}\label{ec:u}
\overline{u}_r(\vb{p})\gamma^\mu m u_s(\vb{p})=\overline{u}_r(\vb{p})p^\mu u_s(\vb{p}).
\end{equation}
Repitiendo este proceso al pie de la letra se tiene
\begin{equation}
0=\overline{v}_r(\vb{p'})\gamma^\mu(\slashed{p}+m)v_s(\vb{p})=\overline{v}_r(\vb{p'})\gamma^\mu\gamma^\nu p_\nu v_s(\vb{p})+\overline{v}_r(\vb{p'})\gamma^\mu m v_s(\vb{p}),
\end{equation}
\begin{equation}
0=\overline{v}_r(\vb{p'})(\slashed{p}'+m)\gamma^\mu v_s(\vb{p})=\overline{v}_r(\vb{p'})\gamma^\nu\gamma^\mu p'_\nu v_s(\vb{p})+\overline{v}_r(\vb{p'})\gamma^\mu m v_s(\vb{p}),
\end{equation}
\begin{align}\label{ec:v_original}
\begin{split}
0=&\overline{v}_r(\vb{p'})\gamma^\mu\gamma^\nu p_\nu v_s(\vb{p})+\overline{v}_r(\vb{p'})\gamma^\mu m v_s(\vb{p})\\
+&\overline{v}_r(\vb{p'})\gamma^\nu\gamma^\mu p'_\nu v_s(\vb{p})+\overline{v}_r(\vb{p'})\gamma^\mu m v_s(\vb{p})\\
=&\overline{v}_r(\vb{p'})\qty(\gamma^\mu\gamma^\nu p_\nu+\gamma^\nu\gamma^\mu p'_\nu) v_s(\vb{p})+2\overline{v}_r(\vb{p'})\gamma^\mu m v_s(\vb{p})\\
=&\overline{v}_r(\vb{p'})2g^{\mu\nu} p_\nu v_s(\vb{p})+2m\overline{v}_r(\vb{p'})\gamma^\mu v_s(\vb{p})
\end{split}
\end{align}
en el caso $\vb{p}=\vb{p'}$ y
\begin{equation}\label{ec:v}
\overline{v}_r(\vb{p'})\gamma^\mu m v_s(\vb{p})=-\overline{v}_r(\vb{p'})p^\mu v_s(\vb{p}).
\end{equation}

Poniendo $\mu=0$ en \eqref{ec:u} se tiene haciendo uso de la normalización (4.49) de \cite{Lahiri2005}
\begin{align}
\begin{split}
2E_p\delta_{rs}m=&mu^\dagger_r(\vb{p})u_s(\vb{p})=mu^\dagger_r(\vb{p})\gamma^0\gamma^0 u_s(\vb{p})=m\overline{u}_r(\vb{p})\gamma^0 u_s(\vb{p})\\
=&p^0\overline{u}_r(\vb{p})u(\vb{p})=E_p\overline{u}_r(\vb{p})u_s(\vb{p}).
\end{split}
\end{align} 
Repitiendo con \eqref{ec:v}
\begin{align}
\begin{split}
2E_p\delta_{rs}m=&mv^\dagger_r(\vb{p})v_s(\vb{p})=mv^\dagger_r(\vb{p})\gamma^0\gamma^0 v_s(\vb{p})=m\overline{v}_r(\vb{p})\gamma^0 v_s(\vb{p})\\
=&-p^0\overline{v}_r(\vb{p})v(\vb{p})=-E_p\overline{v}_r(\vb{p})v_s(\vb{p}).
\end{split}
\end{align} 
Por lo tanto, asumiendo que $E_p\neq 0$, se tiene
\begin{equation}
\overline{u}_r(\vb{p})u_s(\vb{p})=-\overline{v}_r(\vb{p})v_s(\vb{p})=2m\delta_{rs}.
\end{equation}  

\item[4.9] Note que haciendo uso de la relaciones de conmutación de las matrices de Dirac se tiene 
\begin{align}
\begin{split}
&(p+p')^\mu-i\sigma^{\mu\nu}q_\nu\\
=&p^\mu+{p'}^\mu-i\frac{i}{2}[\gamma^\mu,\gamma^\nu]_{-}(p_\nu-p'_\nu)\\
=&p^\mu+{p'}^\mu+\frac{1}{2}(\gamma^\mu\gamma^\nu-\gamma^\nu\gamma^\mu)(p_\nu-p'_\nu)\\
=&p^\mu+{p'}^\mu+\frac{1}{2}(\gamma^\mu\gamma^\nu p_\nu -\gamma^\mu\gamma^\nu p'_\nu - \gamma^\nu\gamma^\mu p_\nu +\gamma^\nu\gamma^\mu p'_\nu)\\
=&p^\mu+{p'}^\mu\\
&+\frac{1}{2}(\gamma^\mu\gamma^\nu p_\nu-2g^{\mu\nu}p'_\nu+\gamma^\nu\gamma^\mu p'_\nu-2g^{\nu\mu}p_\nu+ \gamma^\mu\gamma^\nu p_\nu +\gamma^\nu\gamma^\mu p'_\nu)\\
=&p^\mu+{p'}^\mu+\frac{1}{2}(2\gamma^\mu\gamma^\nu p_\nu-2p'^\mu+2\gamma^\nu\gamma^\mu p'_\nu-2p^\mu)\\
=&\gamma^\mu\gamma^\nu p_\nu+\gamma^\nu\gamma^\mu p'_\nu. 
\end{split}
\end{align}
Por lo tanto, comparando con las ecuaciones \eqref{ec:u_original} y \eqref{ec:v_original} se obtiene
\begin{align}
\begin{split}
0=&\overline{u}_r(\vb{p'})((p+p')^\mu-i\sigma^{\mu\nu}q_\nu)u_s(\vb{p})-2m\overline{u}_r(\vb{p'})\gamma^\mu u_s(\vb{p})\\
0=&\overline{v}_r(\vb{p'})((p+p')^\mu-i\sigma^{\mu\nu}q_\nu)v_s(\vb{p})+2m\overline{v}_r(\vb{p'})\gamma^\mu v_s(\vb{p}).
\end{split}
\end{align}
Dividiendo por $2m$ se obtienen las identidades de Gordon
\begin{align}
\begin{split}
\overline{u}_r(\vb{p'})\gamma^\mu u_s(\vb{p})=&\frac{1}{2m}\overline{u}_r(\vb{p'})((p+p')^\mu-i\sigma^{\mu\nu}q_\nu)u_s(\vb{p})\\
\overline{v}_r(\vb{p'})\gamma^\mu v_s(\vb{p})=&-\frac{1}{2m}\overline{v}_r(\vb{p'})((p+p')^\mu-i\sigma^{\mu\nu}q_\nu)v_s(\vb{p}).
\end{split}
\end{align}

\item[4.10] Como en el ejercicio 4.8, multiplicando a izquierda por $\overline{u}_r(\vb{p})\gamma^\mu$ a la ecuación (4.46) de \cite{Lahiri2005} y por la derecha por $\gamma^\mu v_s(\vb{p})$ a (4.48) de \cite{Lahiri2005} se obtiene
\begin{align}
\begin{split}
\overline{u}_r(\vb{p})\gamma^\mu(\slashed{p}+m)v_s(\vb{p})=&0\\
\overline{u}_r(\vb{p})(\slashed{p}-m)\gamma^\mu v_s(\vb{p})=&0.
\end{split}
\end{align}
Al sumar estas ecuaciones se concluye
\begin{align}
\begin{split}
0=&\overline{u}_r(\vb{p})\gamma^\mu(\slashed{p}+m)v_s(\vb{p})+\overline{u}_r(\vb{p})(\slashed{p}-m)\gamma^\mu v_s(\vb{p})\\
=&\overline{u}_r(\vb{p})(\gamma^\mu\slashed{p}+\slashed{p}\gamma^\mu)v_s(\vb{p})=\overline{u}_r(\vb{p})(\gamma^\mu\gamma^\nu p_\nu+\gamma^\nu\gamma^\mu p_\nu)v_s(\vb{p})\\
=&\overline{u}_r(\vb{p})(\gamma^\mu\gamma^\nu+\gamma^\nu\gamma^\mu)p_\nu v_s(\vb{p})=\overline{u}_r(\vb{p})2g^{\mu\nu}p_\nu v_s(\vb{p})\\
=&2p^\mu\overline{u}_r(\vb{p})v_s(\vb{p}),
\end{split}
\end{align}
es decir,
\begin{equation}\label{ec:ortogonalidad_1}
0=\overline{u}_r(\vb{p})v_s(\vb{p}).
\end{equation}
Conjugando se tiene
\begin{align}\label{ec:ortogonalidad_2}
\begin{split}
0=&(\overline{u}_r(\vb{p})v_s(\vb{p}))^\dagger=(u_r^\dagger(\vb{p})\gamma^0v_s(\vb{p}))^\dagger=v_s(\vb{p})^\dagger\gamma^0 u_r(\vb{p})\\
=&\overline{v}_s(\vb{p})u_r(\vb{p}).
\end{split}
\end{align}

Note que debido a la ecuación de Dirac (4.46) de \cite{Lahiri2005}
\begin{align}
\begin{split}
(\slashed{p}+m)u_r(\vb{p})=&(\slashed{p}-m+2m)u_r(\vb{p})=2mu_r(\vb{p})\\
(\slashed{p}+m)v_r(\vb{p})=&0\\
(\slashed{p}-m)u_r(\vb{p})=&0\\
(\slashed{p}-m)v_r(\vb{p})=&(\slashed{p}+m-2m)v_r(\vb{p})=-2mv_r(\vb{p}).
\end{split}
\end{align}
Por el otro lado, haciendo uso de las relaciones de normalización halladas en el ejercicio 4.8 y las relaciones \eqref{ec:ortogonalidad_1} y \eqref{ec:ortogonalidad_2}
\begin{align}
\begin{split}
\sum_s u_s(\vb{p})\overline{u}_s(\vb{p})u_r(\vb{p})=&\sum_s u_s(\vb{p})2m\delta_{sr}=2mu_r(\vb{p})\\
\sum_s u_s(\vb{p})\overline{u}_s(\vb{p})v_r(\vb{p})=&0\\
\sum_s v_s(\vb{p})\overline{v}_s(\vb{p})u_r(\vb{p})=&0\\
\sum_s v_s(\vb{p})\overline{v}_s(\vb{p})v_r(\vb{p})=&\sum_s v_s(\vb{p})(-2m\delta_{sr})=-2mv_r(\vb{p}).
\end{split}
\end{align}
Ya que las matrices coinciden en una base, por extensión lineal deben ser iguales
\begin{align}
\begin{split}
\sum_s u_s(\vb{p})\overline{u}_s(\vb{p})=&\slashed{p}+m\\
\sum_s v_s(\vb{p})\overline{v}_s(\vb{p})=&\slashed{p}-m.
\end{split}
\end{align}

\item[4.24] Se tiene con la ecuación (A.32) de \cite{Lahiri2005} que
\begin{align}
\begin{split}
W^\mu W_\mu=&\frac{1}{4}\epsilon^{\mu\nu\lambda\rho}\epsilon_{\mu\nu'\lambda'\rho'}P_\nu J_{\lambda\rho}P^{\nu'} J^{\lambda'\rho'}\\
=&-\frac{1}{4}(\delta^\nu_{\nu'}\delta^\lambda_{\lambda'}\delta^\rho_{\rho'}+\delta^\nu_{\lambda'}\delta^\lambda_{\rho'}\delta^\rho_{\nu'}+\delta^\nu_{\rho'}\delta^\lambda_{\nu'}\delta^\rho_{\lambda'}\\
&-\delta^\nu_{\lambda'}\delta^\lambda_{\nu'}\delta^\rho_{\rho'}-\delta^\nu_{\rho'}\delta^\lambda_{\lambda'}\delta^\rho_{\nu'}-\delta^\nu_{\nu'}\delta^\lambda_{\rho'}\delta^\rho_{\lambda'})P_\nu J_{\lambda\rho}P^{\nu'} J^{\lambda'\rho'}\\
=&-\frac{1}{4}P_\nu J_{\lambda\rho}(P^{\nu} J^{\lambda\rho}+P^{\rho} J^{\nu\lambda}+P^{\lambda} J^{\rho\nu}\\
&-P^{\lambda} J^{\nu\rho}-P^{\rho} J^{\lambda\nu}-P^{\nu} J^{\rho\lambda})\\
=&-\frac{1}{4}P_\nu J_{\lambda\rho}(P^{\nu} (J^{\lambda\rho}-J^{\rho\lambda})+P^{\rho} (J^{\nu\lambda}-J^{\lambda\nu})\\
&+P^{\lambda} (J^{\rho\nu}-J^{\nu\rho}))
\end{split}
\end{align}
Note las siguientes propiedades de antisimetría
\begin{align}
\begin{split}
\sigma_{\mu\nu}=&\frac{i}{2}[\gamma^\mu,\gamma^\nu]_-=-\frac{i}{2}[\gamma^\nu,\gamma^\mu]_-=-\sigma^{\nu\mu}\\
J_{\mu\nu}=&i(x_\mu\partial_\nu-x_\nu\partial_\mu)+\frac{1}{2}\sigma_{\mu\nu}=-i(x_\nu\partial_\mu-x_\mu\partial_\nu)-\frac{1}{2} \sigma_{\nu\mu}=-J_{\nu\mu}.
\end{split}
\end{align}
Por lo tanto
\begin{align}
\begin{split}
W^\mu W_\mu=&-\frac{1}{2}P_\nu J_{\lambda\rho}(P^{\nu} J^{\lambda\rho}+P^{\rho} J^{\nu\lambda}+P^{\lambda} J^{\rho\nu})
\end{split}
\end{align}
Ahora bien, note que $W^\mu W_\mu$ es un escalar y por lo tanto no depende del sistema de referencia en el que se evalue. Ya que la fórmula (4.95) de \cite{Lahiri2005} solo tiene sentido para partículas masivas, asumimos que nuestra partícula tiene $m\neq 0$. Por lo tanto en el sistema de reposo de la partícula las componentes espaciales del momento se anulan y
\begin{align}
\begin{split}
W^\mu W_\mu=&-\frac{1}{2}P_0 J_{\lambda\rho}(P^0 J^{\lambda\rho}+P^{\rho} J^{0\lambda}+P^{\lambda} J^{\rho 0})\\
=&-\frac{1}{2}(P_0J_{\lambda\rho}P^0J^{\lambda\rho}+P_0J_{\lambda\rho}P^\rho J^{0\lambda}+P_0J_{\lambda\rho}P^\lambda J^{\rho0})\\
=&-\frac{1}{2}(P_0J_{\lambda\rho}P^0J^{\lambda\rho}+P_0J_{\lambda0}P^0 J^{0\lambda}+P_0J_{0\rho}P^0 J^{\rho0})\\
=&-\frac{1}{2}(P_0J_{\lambda\rho}P^0J^{\lambda\rho}+P_0J_{\lambda0}P^0 J^{0\lambda}+P_0J_{\rho 0}P^0 J^{0\rho})\\
=&-\frac{1}{2}(P_0J_{\lambda\rho}P^0J^{\lambda\rho}+2P_0J_{\lambda0}P^0 J^{0\lambda}.)
\end{split}
\end{align} 
Más aún, en el sistema de reposo el momento angular orbital es nulo y por lo tanto
\begin{equation}
J_{\mu\nu}=\frac{1}{2}\sigma_{\mu\nu}.
\end{equation}
Ya que actuan en espacios distintos se tiene
\begin{equation}
[P^\mu,\sigma^{\nu\lambda}]_-=0.
\end{equation}
Además, en el centro de masa el cuadrado de la energía es $m^2$, es decir $P^0P_0=m^2$. Se concluye
\begin{equation}
W^\mu W_\mu=-\frac{1}{8}m^2(\sigma_{\lambda\rho}\sigma^{\lambda\rho}+2\sigma_{\lambda0}\sigma^{0\lambda}).
\end{equation}
Note las siguientes identidades
\begin{align}
\begin{split}
\gamma^0\gamma_0=&\gamma_0\gamma_0=1,\\
\gamma^\mu\gamma_\mu=&g^{\mu\nu}\gamma_\nu\gamma_\mu=\frac{1}{2}(g^{\mu\nu}+g^{\nu\mu})\gamma_\nu\gamma_\mu=\frac{1}{2}(g^{\mu\nu}\gamma_\nu\gamma_\mu+g^{\nu\mu}\gamma_\nu\gamma_\mu)\\
=&\frac{1}{2}(g^{\mu\nu}\gamma_\nu\gamma_\mu+g^{\mu\nu}\gamma_\mu\gamma_\nu)=\frac{1}{2}g^{\mu\nu}[\gamma_\mu,\gamma_\nu]_-=\frac{1}{2}g^{\mu\nu}2g_{\mu\nu}\\
=&\delta^\mu_\mu=4,\\
\gamma^\mu\gamma^\nu\gamma_\mu=&(2g^{\mu\nu}-\gamma^\nu\gamma^\mu)\gamma_\mu=2\gamma^\nu-4\gamma^\nu=-2\gamma^\nu.
\end{split}
\end{align}
Por lo tanto se concluye que 
\begin{align}
\begin{split}
\sigma_{\lambda\rho}\sigma^{\lambda\rho}=&-\frac{1}{4}(\gamma_\lambda\gamma_\rho-\gamma_\rho\gamma_\lambda)(\gamma^\lambda\gamma^\rho-\gamma^\rho\gamma^\lambda)\\
=&-\frac{1}{4}(\gamma_\lambda\gamma_\rho\gamma^\lambda\gamma^\rho-\gamma_\lambda\gamma_\rho\gamma^\rho\gamma^\lambda-\gamma_\rho\gamma_\lambda\gamma^\lambda\gamma^\rho+\gamma_\rho\gamma_\lambda\gamma^\rho\gamma^\lambda)\\
=&-\frac{1}{4}(\gamma_\lambda(-2\gamma^\lambda)-4\gamma_\lambda\gamma^\lambda-4\gamma_\rho\gamma^\rho+\gamma_\rho(-2\gamma^\rho))\\
=&-\frac{1}{4}(-8-16-16-8)=\frac{48}{4}=12\\
\sigma_{\lambda 0}\sigma^{0\lambda}=&-\frac{1}{4}(\gamma_\lambda\gamma_0-\gamma_0\gamma_\lambda)(\gamma^0\gamma^\lambda-\gamma^\lambda\gamma^0)\\
=&-\frac{1}{4}(\gamma_\lambda\gamma_0\gamma^0\gamma^\lambda-\gamma_\lambda\gamma_0\gamma^\lambda\gamma^0-\gamma_0\gamma_\lambda\gamma^0\gamma^\lambda+\gamma_0\gamma_\lambda\gamma^\lambda\gamma^0)\\
=&-\frac{1}{4}(\gamma_\lambda\gamma^\lambda-(-2\gamma_0)\gamma^0-\gamma_0(-2\gamma^0)+4\gamma_0\gamma^0)\\
=&-\frac{1}{4}(4+2+2+4)=-\frac{12}{4}=-3.
\end{split}
\end{align}
Este invariante entonces satisface
\begin{align}
\begin{split}
-m^2s(s+1)=W^\mu W_\mu=-\frac{1}{8}m^2(12-2\times3)=-\frac{6}{8}m^2=-\frac{3}{4}m^2.
\end{split}
\end{align}
Esto nos lleva a la ecuación cuadrática $s^2+s-\frac{3}{4}=0$ cuyas soluciones son
\begin{equation}
s=\frac{-1\pm\sqrt{1+3}}{2}=\frac{-1\pm 2}{2}=\begin{cases}
\frac{1}{2}\\
-\frac{3}{2}.
\end{cases}
\end{equation}
En particular, ya que el espín total es positivo, se concluye que
\begin{equation}
s=\frac{1}{2}.
\end{equation}

\end{enumerate}

%\bibliography{/Users/ivan/Documents/Bib_Files/library}
\bibliography{/home/UANDES/im.burbano10/Documents/library}
\bibliographystyle{ieeetr}

\end{document}
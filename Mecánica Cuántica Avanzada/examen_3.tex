\documentclass{article}

\usepackage[utf8]{inputenc}
\usepackage[spanish]{babel}
\usepackage{enumerate}
\usepackage{physics}
\usepackage{amssymb}

\title{Mecánica Cuántica Avanzada:\\
Examen 3}
\author{Iván Mauricio Burbano Aldana\\
201423205}

\begin{document}

\maketitle

\section*{Problema 2}

\begin{enumerate}[(a)]

\item Se tiene que
\begin{align}
\begin{split}
\pdv{\mathcal{L}}{\partial_\mu\phi^\dagger}=&\pdv{\partial_\mu\phi^\dagger}((\partial_\nu\phi^\dagger)(\partial^\nu\phi))=\partial^\nu\phi\pdv{\partial_\mu\phi^\dagger}(\partial_\nu\phi^\dagger)=\partial^\nu\phi\delta^\mu_\nu=\partial^\mu\phi,\\
\pdv{\mathcal{L}}{\phi^\dagger}=&-m^2\phi.
\end{split}
\end{align}
Por lo tanto, las ecuaciones de Euler-Lagrange entregan la de Klein-Gordon
\begin{equation}
0=\partial_\mu\pdv{\mathcal{L}}{\partial_\mu\phi^\dagger}-\pdv{\mathcal{L}}{\phi^\dagger}=\partial_\mu\partial^\mu\phi+m^2\phi=(\Box+m^2)\phi.
\end{equation}

\item Bajo esta transformación $\phi\mapsto (e^{-iq\theta}\phi)^\dagger=e^{iq\theta}\phi^\dagger$ y el Lagrangiano es
\begin{align}
\begin{split}
\mathcal{L}(e^{-iq\theta}\phi,e^{iq\theta}\phi^\dagger)=&(\partial_\mu(e^{iq\theta}\phi^\dagger))(\partial^\mu(e^{-iq\theta}\phi))-m^2e^{iq\theta}\phi^\dagger e^{-iq\theta}\phi\\
=&(e^{iq\theta}\partial_\mu\phi^\dagger)(e^{-iq\theta}\partial^\mu\phi)-m^2\phi^\dagger\phi\\
=&(\partial_\mu\phi^\dagger)(\partial^\mu\phi)-m^2\phi^\dagger\phi=\mathcal{L}(\phi,\phi^\dagger).
\end{split}
\end{align}
Se concluye que en efecto el Lagrangiano es invariante.

\item Asumiendo $q\theta$ como pequeño se tiene $\phi\mapsto\phi-iq\theta\phi$ y $\phi^\dagger\mapsto\phi^\dagger+iq\theta\phi^\dagger$. Ya que el Lagrangiano es invariante, la corriente de Noether obtenida es particularmente sencilla
\begin{equation}\label{ec:corriente_escalar}
j^\mu=\pdv{\mathcal{L}}{\partial_\mu\phi} (-iq\theta\phi)+\pdv{\mathcal{L}}{\partial_\mu\phi^\dagger} (iq\theta\phi^\dagger)=iq\theta\qty((\partial^\mu\phi)\phi^\dagger-\qty(\partial^\mu\phi^\dagger)\phi).
\end{equation}
Esto nos va a ser util una vez cuanticemos el campo. Ahora bien, a partir de la ecuación del enunciado
\begin{align}
\begin{split}
\phi^\dagger(x)=&\int\frac{\dd[3]{\vb{p}}}{(2\pi)^{3/2}}\frac{1}{\sqrt{2E_{\vb{p}}}}\qty(a^\dagger(\vb{p})e^{ip\vdot x}+\hat{a}(\vb{p})e^{-ip\vdot x}),\\
\partial^\mu\phi(x)=&\int\frac{\dd[3]{\vb{p}}}{(2\pi)^{3/2}}\frac{1}{\sqrt{2E_{\vb{p}}}}ip^\mu(-a(p)e^{-ip\vdot x}+\hat{a}^\dagger(p)e^{ip\vdot x}),\\
\partial^\mu\phi^\dagger(x)=&\int\frac{\dd[3]{\vb{p}}}{(2\pi)^{3/2}}\frac{1}{\sqrt{2E_{\vb{p}}}}ip^\mu\qty(a^\dagger(\vb{p})e^{ip\vdot x}-\hat{a}(\vb{p})e^{-ip\vdot x}).
\end{split}
\end{align}
Entonces, en terminos de los operadores de creación y destrucción se obtiene
\begin{align}
\begin{split}
(\partial^\mu\phi(x))\phi^\dagger(x)=&\int\frac{\dd[3]{\vb{p}}\dd[3]{\vb{q}}}{(2\pi)^3}\frac{1}{2\sqrt{E_{\vb{p}}E_{\vb{q}}}}ip^\mu\\
&\left(-a(\vb{p})a^\dagger(\vb{q})e^{-i(p-q)\vdot x}+\hat{a}^\dagger(\vb{p})\hat{a}(\vb{q})e^{i(p-q)\vdot x}\right.\\
&\left.-a(\vb{p})\hat{a}(\vb{q})e^{-i(p+q)\vdot x}+\hat{a}^\dagger(\vb{p})a^\dagger(\vb{q})e^{i(p+q)\vdot x}\right),\\
(\partial^\mu\phi^\dagger(x))\phi(x)=&\int\frac{\dd[3]{\vb{p}}\dd[3]{\vb{q}}}{(2\pi)^3}\frac{1}{2\sqrt{E_{\vb{p}}E_{\vb{q}}}}ip^\mu\\
&\left(a^\dagger(\vb{p})a(\vb{q})e^{i(p-q)\vdot x}-\hat{a}(\vb{p})\hat{a}^\dagger(\vb{q})e^{-i(p-q)\vdot x}\right.\\
&\left.-\hat{a}(\vb{p})a(\vb{q})e^{-i(p+q)\vdot x}+a^\dagger(\vb{p})\hat{a}^\dagger(\vb{q})e^{i(p+q)\vdot x}\right).
\end{split}
\end{align}
Al restar estos términos se pueden factorizar las exponenciales comunes de manera que se obtiene
\begin{align}
\begin{split}
j^\mu(x)=&iq\theta\int\frac{\dd[3]{\vb{p}}\dd[3]{\vb{q}}}{(2\pi)^3}\frac{1}{2\sqrt{E_{\vb{p}}E_{\vb{q}}}}ip^\mu\\
&\left(\qty(\hat{a}(\vb{p})\hat{a}^\dagger(\vb{q})-a(\vb{p})a^\dagger(\vb{q}))e^{-i(p-q)\vdot x}+\right.\\
&\left.\qty(\hat{a}^\dagger(\vb{p})\hat{a}(\vb{q})-a^\dagger(\vb{p})a(\vb{q}))e^{i(p-q)\vdot x}\right.\\
&\left.\qty(\hat{a}(\vb{p})a(\vb{q})-a(\vb{p})\hat{a}(\vb{q}))e^{-i(p+q)\vdot x}+\right.\\
&\left.\qty(\hat{a}^\dagger(\vb{p})a^\dagger(\vb{q})-a^\dagger(\vb{p})\hat{a}^\dagger(\vb{q}))e^{i(p+q)\vdot x}\right).
\end{split}
\end{align}
Para el cálculo de la carga conservada note que 
\begin{align}
\begin{split}
\int\dd[3]{\vb{x}}e^{\pm i(p\pm q)\vdot x}=&\int\dd[3]{\vb{x}}e^{\pm i(E_{\vb{p}}\pm E_{\vb{q}})t}e^{\mp i(\vb{p}\pm\vb{q})\vdot\vb{x}}\\
=&(2\pi)^3e^{\pm i(E_{\vb{p}}\pm E_{\vb{q}})t}\delta^{(3)}(\vb{p}\pm\vb{q}).
\end{split}
\end{align}
Por lo tanto
\begin{align}
\begin{split}
\int\dd[3]{\vb{x}}j^\mu(x)=&iq\theta\int\dd[3]{\vb{p}}\dd[3]{\vb{q}}\frac{1}{2\sqrt{E_{\vb{p}}E_{\vb{q}}}}ip^\mu\\
&\left(\qty(\hat{a}(\vb{p})\hat{a}^\dagger(\vb{q})-a(\vb{p})a^\dagger(\vb{q}))e^{-i(E_{\vb{p}}-E_{\vb{q}})t}\delta^{(3)}(\vb{p}-\vb{q})+\right.\\
&\left.\qty(\hat{a}^\dagger(\vb{p})\hat{a}(\vb{q})-a^\dagger(\vb{p})a(\vb{q}))e^{i(E_{\vb{p}}-E_{\vb{q}})t}\delta^{(3)}(\vb{p}-\vb{q})\right.\\
&\left.\qty(\hat{a}(\vb{p})a(\vb{q})-a(\vb{p})\hat{a}(\vb{q}))e^{-i(E_{\vb{p}}+E_{\vb{q}})t}\delta^{(3)}(\vb{p}+\vb{q})+\right.\\
&\left.\qty(\hat{a}^\dagger(\vb{p})a^\dagger(\vb{q})-a^\dagger(\vb{p})\hat{a}^\dagger(\vb{q}))e^{i(E_{\vb{p}}+E_{\vb{q}})t}\delta^{(3)}(\vb{p}+\vb{q})\right).
\end{split}
\end{align}
Más aún, ya que $E_{\vb{p}}=\sqrt{m^2+\vb{p}^2}=\sqrt{m^2+(-\vb{p})^2}=E_{-\vb{p}}$ el coeficiente de cada termino va a ser igual para $\vb{q}=\pm\vb{p}$ y podemos realizar la integración sobre $\vb{q}$
\begin{align}
\begin{split}
\int\dd[3]{\vb{x}}j^\mu(x)=&iq\theta\int\dd[3]{\vb{p}}\frac{1}{2E_{\vb{p}}}ip^\mu\\
&\left(\qty(\hat{a}(\vb{p})\hat{a}^\dagger(\vb{p})-a(\vb{p})a^\dagger(\vb{p}))+\qty(\hat{a}^\dagger(\vb{p})\hat{a}(\vb{p})-a^\dagger(\vb{p})a(\vb{p}))\right.\\
&\left.\qty(\hat{a}(\vb{p})a(-\vb{p})-a(\vb{p})\hat{a}(-\vb{p}))e^{-2iE_{\vb{p}}t}+\right.\\
&\left.\qty(\hat{a}^\dagger(\vb{p})a^\dagger(-\vb{p})-a^\dagger(\vb{p})\hat{a}^\dagger(-\vb{p}))e^{2iE_{\vb{p}}t}\right).
\end{split}
\end{align}
En particular, la carga conservada se obtiene al poner $\mu=0$
\begin{align}
\begin{split}
Q=&iq\theta\int\dd[3]{\vb{p}}\frac{i}{2}\\
&\left(\qty(\hat{a}(\vb{p})\hat{a}^\dagger(\vb{p})-a(\vb{p})a^\dagger(\vb{p}))+\qty(\hat{a}^\dagger(\vb{p})\hat{a}(\vb{p})-a^\dagger(\vb{p})a(\vb{p}))\right.\\
&\left.\qty(\hat{a}(\vb{p})a(-\vb{p})-a(\vb{p})\hat{a}(-\vb{p}))e^{-2iE_{\vb{p}}t}+\right.\\
&\left.\qty(\hat{a}^\dagger(\vb{p})a^\dagger(-\vb{p})-a^\dagger(\vb{p})\hat{a}^\dagger(-\vb{p}))e^{2iE_{\vb{p}}t}\right).
\end{split}
\end{align}
Ya que $[a(\vb{p}),\hat{a}(\vb{q})]=[a^\dagger(\vb{p}),\hat{a}^\dagger(\vb{q})]=0$ esto se puede reescribir de manera más sugestiva
\begin{align}
\begin{split}
Q=&iq\theta\int\dd[3]{\vb{p}}\frac{i}{2}\\
&\left(\qty(\hat{a}(\vb{p})\hat{a}^\dagger(\vb{p})-a(\vb{p})a^\dagger(\vb{p}))+\qty(\hat{a}^\dagger(\vb{p})\hat{a}(\vb{p})-a^\dagger(\vb{p})a(\vb{p}))\right.\\
&\left.\qty(\hat{a}(\vb{p})a(-\vb{p})-\hat{a}(-\vb{p})a(\vb{p}))e^{-2iE_{\vb{p}}t}+\right.\\
&\left.\qty(\hat{a}^\dagger(\vb{p})a^\dagger(-\vb{p})-\hat{a}^\dagger(-\vb{p})a^\dagger(\vb{p}))e^{2iE_{\vb{p}}t}\right).
\end{split}
\end{align}
Esta forma sugiere considerar el cambio de variable $\vb{p}\mapsto -\vb{p}$ bajo el cual la integral se mantiene invariante debido a que la energía lo hace y el cambio en los diferenciales se compensa con el de la orientación de la región de integración
\begin{align}
\begin{split}
\int_{\mathbb{R}^3}\dd[3]{\vb{p}}\hat{a}(-\vb{p})a(\vb{p})e^{-2iE_{\vb{p}}t}=&\int_{-\mathbb{R}^3}\dd[3](-\vb{p})\hat{a}(\vb{p})a(-\vb{p})e^{-2iE_{-\vb{p}}t}\\
=&(-1)^3\int_{-\mathbb{R}^3}\dd[3]{\vb{p}}\hat{a}(\vb{p})a(-\vb{p})e^{-2iE_{\vb{p}}t}\\
=&\int_{\mathbb{R}^3}\dd[3]{\vb{p}}\hat{a}(\vb{p})a(-\vb{p})e^{-2iE_{\vb{p}}t},\\
\int_{\mathbb{R}^3}\dd[3]{\vb{p}}\hat{a}^\dagger(-\vb{p})a^\dagger(\vb{p})e^{-2iE_{\vb{p}}t}=&\int_{-\mathbb{R}^3}\dd[3](-\vb{p})\hat{a}^\dagger(\vb{p})a^\dagger(-\vb{p})e^{-2iE_{-\vb{p}}t}\\
=&(-1)^3\int_{-\mathbb{R}^3}\dd[3]{\vb{p}}\hat{a}^\dagger(\vb{p})a^\dagger(-\vb{p})e^{-2iE_{\vb{p}}t}\\
=&\int_{\mathbb{R}^3}\dd[3]{\vb{p}}\hat{a}^\dagger(\vb{p})a^\dagger(-\vb{p})e^{-2iE_{\vb{p}}t}.
\end{split}
\end{align}
Entonces los coeficientes de las exponenciales se anulan de manera que al sacar el factor $i/2$ de la integral obtenemos
\begin{align}
\begin{split}
Q=&\frac{q}{2}\theta\int\dd[3]{\vb{p}}\\
&\qty(\qty(a(\vb{p})a^\dagger(\vb{p})-\hat{a}(\vb{p})\hat{a}^\dagger(\vb{p}))+\qty(a^\dagger(\vb{p})a(\vb{p})-\hat{a}^\dagger(\vb{p})\hat{a}(\vb{p})))
\end{split}
\end{align}
Ya que $[a(\vb{p}),a^\dagger(\vb{q})]=\delta^{(3)}(\vb{p}-\vb{q})=[\hat{a}(\vb{p}),\hat{a}^\dagger(\vb{q})]$ se obtiene que
\begin{align}
\begin{split}
&a(\vb{p})a^\dagger(\vb{q})-\hat{a}(\vb{p})\hat{a}^\dagger(\vb{q})\\
=&a^\dagger(\vb{q})a(\vb{p})+[a(\vb{p}),a^\dagger(\vb{q})]-\hat{a}^\dagger(\vb{q})\hat{a}(\vb{p})-[\hat{a}(\vb{p}),\hat{a}^\dagger(\vb{q})]\\
=&a^\dagger(\vb{q})a(\vb{p})-\hat{a}^\dagger(\vb{q})\hat{a}(\vb{p}).
\end{split}
\end{align}
Poniendo $\vb{q}=\vb{p}$ vemos que los dos términos de la suma son iguales y se obtiene una carga conservada
\begin{equation}
Q=q\theta\int\dd[3]{\vb{p}}\qty(a^\dagger(\vb{p})a(\vb{p})-\hat{a}^\dagger(\vb{p})\hat{a}(\vb{p})).
\end{equation}
Si una cantidad se conserva es claro que cualquier múltiplo de ella también. Ya que el parámetro de la transformación $q\theta$ no tiene una interpretación clara por el momento, podemos redefinir la carga de Noether a 
\begin{equation}
Q=\int\dd[3]{\vb{p}}\qty(a^\dagger(\vb{p})a(\vb{p})-\hat{a}^\dagger(\vb{p})\hat{a}(\vb{p})).
\end{equation}
Podemos definir los operadores
\begin{align}
\begin{split}
\mathcal{N}_a:=&a^\dagger(\vb{p})a(\vb{p}),\\
\mathcal{N}_{\hat{a}}:=&\hat{a}^\dagger(\vb{p})\hat{a}(\vb{p}).
\end{split}
\end{align}
de manera que
\begin{equation}
Q=\int\dd[3]{\vb{p}}\qty(\mathcal{N}_a-\mathcal{N}_{\hat{a}}).
\end{equation}
En la interpretación de partículas esto corresponde a que estas se crean de a pares.

\end{enumerate}

\section*{Problema 3}

\end{document}
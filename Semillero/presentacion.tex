\documentclass{beamer}
\usetheme{Boadilla}

\usepackage[utf8]{inputenc}

\title{What is a Physical Theory?}
\subtitle{An Introduction to the Algebraic Formulation of Physics}
\author[I.M. Burbano]{Iván Mauricio Burbano Aldana}
\institute[Uniandes]{Universidad de los Andes}

\begin{document}

\frame{\titlepage}

\begin{frame}

\frametitle{What is a State?}

The state of a physical system is the minimum amount of information required to predict the outcome of an observation. 

\Huge This is provisionary!!!

\end{frame}

\begin{frame}

\frametitle{Classical states}

Let us recall:

\begin{block}{Newton's Second Law}
The alteration of motion is ever proportional to the motive force impressed; and is made in the direction of the right line in which that force is impressed. — If a force generates a motion, a double force will generate double the motion, a triple force triple the motion, whether that force be impressed altogether and at once, or gradually and successively. And this motion (being always directed the same way with the generating force), if the body moved before, is added to or subtracted from the former motion, according as they directly conspire with or are directly contrary to each other; or obliquely joined, when they are oblique, so as to produce a new motion compounded from the determination of both.
\end{block}

\end{frame}

\begin{frame}

In a mothern formulation

\begin{block}{Newton's Second Law (Modern Formulation)}

\end{block}

\end{frame}

\end{document}
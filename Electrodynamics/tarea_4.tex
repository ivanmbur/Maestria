\documentclass{article}

\usepackage[utf8]{inputenc}
\usepackage{physics}
\usepackage{enumerate}

\title{Electrodynamics: Homework 4}
\author{Iván Mauricio Burbano Aldana}

\begin{document}

\maketitle

\begin{enumerate}

\item We have
\begin{equation}
T^{i0}=F^i_\gamma F^{0\gamma}-\frac{1}{4}\eta^{i0}F_{\gamma\delta}F^{\gamma\delta}.
\end{equation}
Given that $\eta^{\mu\nu}=0$ if $\mu\neq\nu$ and $i\in\{1,2,3\}$ we conclude
\begin{equation}
T^{i0}=F^i_\gamma F^{0\gamma}=\eta_{\rho\gamma}F^{i\rho}F^{0\gamma}
\end{equation}
Remembering that $F^{\mu\nu}$ is antisymmetric and thus that $F^{00}=0$, we notice that
\begin{equation}
T^{i0}=\eta_{\rho j}F^{i\rho}F^{0j}=-\sum_{j=1}^3F^{ij}F^{0j}.
\end{equation}
Recalling that $F^{0j}=E_j$ and $F^{ij}=\sum_{k=1}^3\epsilon_{ijk}B_k$ we have
\begin{equation}
T^{i0}=-\sum_{j=1}^3\sum_{k=1}^3\epsilon_{ijk}B_kE_j=-(\vb{E}\cross\vb{B})_i
\end{equation}

\item \begin{enumerate}[(i)]

\item Indeed, $\mathcal{L}$ is real. This is because $\mathcal{L}_\phi$, $\mathcal{L}_A$ and $\mathcal{L}_{\text{int}}$ are.

\end{enumerate}

\end{enumerate}

\end{document}
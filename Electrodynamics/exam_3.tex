\documentclass{article}

\usepackage[utf8]{inputenc}
\usepackage{enumerate}
\usepackage{physics}
\usepackage{amssymb}

\title{Electrodynamics\\
Third Exam\\
2018-I}
\author{Iván Mauricio Burbano Aldana}

\begin{document}

\maketitle

Throughout this exam I will make use of Einstein's summation convention where repeated indices are summed over. $\{\vu{e}_1,\vu{e}_2,\vu{e}_3\}$ is the canonical basis of $\mathbb{R}^3$.

\begin{enumerate}[(i)]

\item As we have seen in class the potential $A^\mu$ is given by
\begin{equation}
A^\mu(\vb{r},t)=\frac{\mu_0}{4\pi}\int \dd[3]{\vb{r}'}\frac{j^\mu(\vb{r}',t-\|\vb{r}-\vb{r}'\|/c)}{\|\vb{r}-\vb{r}'\|}
\end{equation}
ignoring all background waves, that is, solutions of $\Box A^\mu=0$. Therefore, its temporal Fourier transform, if we assume exists, is given by
\begin{align}
\begin{split}
A^\mu(\vb{r},\omega)=&\int\dd{t}e^{i\omega t}A^\mu(\vb{r},t)\\
=&\frac{\mu_0}{4\pi}\int\dd{t}e^{i\omega t}\int \dd[3]{\vb{r}'}\frac{j^\mu(\vb{r}',t-\|\vb{r}-\vb{r}'\|/c)}{\|\vb{r}-\vb{r}'\|}\\
=&\frac{\mu_0}{4\pi}\int\dd{t}\int\dd[3]{\vb{r}'}\frac{1}{\|\vb{r}-\vb{r}'\|}e^{i\omega t}j^\mu(\vb{r}',t-\|\vb{r}-\vb{r}'\|/c).
\end{split}
\end{align}
Assuming the proper conditions are met to apply Fubini's theorem we have
\begin{align}
\begin{split}
A^\mu(\vb{r},\omega)=&\frac{\mu_0}{4\pi}\int\dd[3]{\vb{r}'}\int\dd{t}\frac{1}{\|\vb{r}-\vb{r}'\|}e^{i\omega t}j^\mu(\vb{r}',t-\|\vb{r}-\vb{r}'\|/c)\\
=&\frac{\mu_0}{4\pi}\int \dd[3]{\vb{r}'}\frac{1}{\|\vb{r}-\vb{r}'\|}\int\dd{t}e^{i\omega t}j^\mu(\vb{r}',t-\|\vb{r}-\vb{r}'\|/c)
\end{split}
\end{align}
Through the change of variables
\begin{align}
\begin{split}
\mathbb{R}&\rightarrow\mathbb{R}\\
t&\mapsto t+\|\vb{r}-\vb{r}'\|/c
\end{split}
\end{align}
which keeps the domain $\mathbb{R}$ invariant we obtain
\begin{align}
\begin{split}
A^\mu(\vb{r},\omega)=&\frac{\mu_0}{4\pi}\int \dd[3]{\vb{r}'}\frac{1}{\|\vb{r}-\vb{r}'\|}\int\dd{t}e^{i\omega(t+\|\vb{r}-\vb{r}'\|/c)}j^\mu(\vb{r}',t)\\
=&\frac{\mu_0}{4\pi}\int \dd[3]{\vb{r}'}\frac{1}{\|\vb{r}-\vb{r}'\|}\int\dd{t}e^{i\omega\|\vb{r}-\vb{r}'\|/c}e^{i\omega t}j^\mu(\vb{r}',t)\\
=&\frac{\mu_0}{4\pi}\int \dd[3]{\vb{r}'}\frac{e^{iK\|\vb{r}-\vb{r}'\|}}{\|\vb{r}-\vb{r}'\|}\int\dd{t}e^{i\omega t}j^\mu(\vb{r}',t).
\end{split}
\end{align}
We now recognize the Fourier transform
\begin{equation}\label{ec:fourier_current}
j^\mu(\vb{r},\omega)=\int\dd{t}e^{i\omega t}j^\mu(\vb{r},t)
\end{equation}
concluding
\begin{equation}
A^\mu(\vb{r},\omega)=\frac{\mu_0}{4\pi}\int \dd[3]{\vb{r}'}\frac{e^{iK\|\vb{r}-\vb{r}'\|}}{\|\vb{r}-\vb{r}'\|}j^\mu(\vb{r}',\omega).
\end{equation}
Separating this equation by remembering that $A^\mu=(\phi/c,\vb{A})$, $j^\mu=(c\rho,\vb{J})$, and $c^2\mu_0=\frac{\mu_0}{\mu_0\epsilon_0}=\frac{1}{\epsilon_0}$ we obtain
\begin{align}\label{ec:fourier_vector_potential}
\begin{split}
\phi(\vb{r},\omega)=&\frac{1}{4\pi\epsilon_0}\int \dd[3]{\vb{r}'}\frac{e^{iK\|\vb{r}-\vb{r}'\|}}{\|\vb{r}-\vb{r}'\|}\rho(\vb{r}',\omega),\\
\vb{A}(\vb{r},\omega)=&\frac{\mu_0}{4\pi}\int \dd[3]{\vb{r}'}\frac{e^{iK\|\vb{r}-\vb{r}'\|}}{\|\vb{r}-\vb{r}'\|}\vb{J}(\vb{r}',\omega).
\end{split}
\end{align}

\item Recall that the law of charge current conservation is given by
\begin{equation}
\partial_\mu j^\mu(\vb{r},t)=0.
\end{equation}
Assuming that the Fourier transformation \eqref{ec:fourier_current} is invertible, we must have
\begin{equation}
j^\mu(\vb{r},t)=\frac{1}{2\pi}\int\dd{\omega}e^{-i\omega t}j^\mu(\vb{r},\omega)
\end{equation}
and
\begin{equation}
0=\frac{1}{2\pi}\partial_\mu\int\dd{\omega}e^{-i\omega t}j^\mu(\vb{r},\omega).
\end{equation}
Assuming that the right conditions are met to interchange differentiation with integration we have
\begin{align}
\begin{split}
0=&\int\dd{\omega}\partial_\mu\qty(e^{-i\omega t}j^\mu(\vb{r},\omega))\\
=&\int\dd{\omega}\qty(\frac{1}{c}\pdv{e^{-i\omega t}}{t}c\rho(\vb{r},\omega)+e^{-i\omega t}\div{\vb{J}}(\vb{r},\omega))\\
=&\int\dd{\omega}\qty(-i\omega e^{-i\omega t}\rho(\vb{r},\omega)+e^{-i\omega t}\div{\vb{J}}(\vb{r},\omega))\\
=&\int\dd{\omega}e^{-i\omega t}\qty(-i\omega\rho(\vb{r},\omega)+\div{\vb{J}}(\vb{r},\omega)).
\end{split}
\end{align}
Under the right conditions the uniqueness theorem is valid and a function is null if and only if is Fourier transform is. We thus conclude that the law of charge current conservation can be expressed as
\begin{equation}\label{ec:fourier_charge_current}
\div{\vb{J}}(\vb{r},\omega)=i\omega\rho(\vb{r},\omega).
\end{equation}

\item We recall the definition of the electric dipole moment and magnetic dipole moment
\begin{align}
\begin{split}
\vb{p}=&\int\dd[3]{\vb{r}}\vb{r}\rho(\vb{r}),\\
\vb{m}=&\frac{1}{2}\int\dd[3]{\vb{r}}\vb{r}\cross\vb{J}(\vb{r}).
\end{split}
\end{align}
By using the product rule of the divergence
\begin{align}
\begin{split}
\int\dd[3]{\vb{r}}\vb{J}(\vb{r},\omega)=&\sum_{i=1}^3\vu{e}_i\int\dd[3]{\vb{r}}\vb{J}(\vb{r},\omega)\vdot\vu{e}_{i}\\
=&\sum_{i=1}^3\vu{e}_i\int\dd[3]{\vb{r}}\vb{J}(\vb{r},\omega)\vdot\grad{x^i}\\
=&\sum_{i=1}^3\vu{e}_i\int\dd[3]{\vb{r}}(\div(x^i\vb{J})(\vb{r},\omega)-x^i\div{\vb{J}}(\vb{r},\omega))\\
=&\sum_{i=1}^3\vu{e}_i\qty(\int\dd[3]{\vb{r}}\div(x^i\vb{J})(\vb{r},\omega)-\int\dd[3]{\vb{r}}x^i\div{\vb{J}}(\vb{r},\omega)).
\end{split}
\end{align}
The first integral vanishes since it is a total differential on a region without a boundary, namely $\mathbb{R}^3$. Therefore, by using \eqref{ec:fourier_charge_current} we obtain
\begin{align}\label{ec:integral_current}
\begin{split}
\int\dd[3]{\vb{r}}\vb{J}(\vb{r},\omega)=&-\sum_{i=1}^3\vu{e}_i\int\dd[3]{\vb{r}}x^i\div{\vb{J}}(\vb{r},\omega)\\
=&-\int\dd[3]{\vb{r}}\sum_{i=1}^3\vu{e}_ix^ii\omega\rho(\vb{r},\omega)=-\int\dd[3]{\vb{r}}\vb{r}i\omega\rho(\vb{r},\omega)\\
=&-i\omega\vb{p}.
\end{split}
\end{align}

In the far zone we have as hinted in the problem sheet
\begin{equation}
\frac{
1}{\|\vb{r}-\vb{r}'\|}\approx\frac{
1}{\|\vb{r}\|-\frac{\vb{r}\vdot\vb{r}'}{\|\vb{r}\|}}=\frac{
1}{r-\frac{\vb{r}\vdot\vb{r}'}{r}}=\frac{1}{r}\frac{1}{1-\frac{\vb{r}\vdot\vb{r}'}{r^2}}
\end{equation}
where $r:=\|\vb{r}\|$ to lighten notation. Given that
\begin{equation}\label{ec:estimate_1}
\left|\frac{\vb{r}\vdot\vb{r}'}{r^2}\right|\leq\frac{rr'}{r^2}=\frac{r'}{r}
\end{equation}
if we assume that the fields are local, that is, that $r'\ll r$ in the region where fields are relevant we may say that under the integral sign $|r'/r|\ll 1$. Therefore, we may employ the geometric series
\begin{equation}
\frac{1}{1-x}=\sum_{n=0}^\infty x^n
\end{equation}
to approximate
\begin{equation}
\frac{
1}{\|\vb{r}-\vb{r}'\|}\approx\frac{1}{r}\qty(1+\frac{\vb{r}\vdot\vb{r}'}{r^2}).
\end{equation}
Similarly, we may stablish the estimate
\begin{equation}
e^{ik\|\vb{r}-\vb{r}'\|}\approx e^{ik\qty(r-\frac{\vb{r}\vdot\vb{r}'}{r})}=e^{ikr}e^{-ik\frac{\vb{r}\vdot\vb{r}'}{r}}=e^{ikr}\sum_{n=0}^\infty\frac{(-ik)^n}{n!}\qty(\frac{\vb{r}\vdot\vb{r}'}{r})^n.
\end{equation}
Thus,
\begin{align}
\begin{split}
\frac{
e^{ik\|\vb{r}-\vb{r}'\|}}{\|\vb{r}-\vb{r}'\|}\approx&\frac{1}{r}\qty(1+\frac{\vb{r}\vdot\vb{r}'}{r^2})e^{ikr}\sum_{n=0}^\infty\frac{(-ik)^n}{n!}\qty(\frac{\vb{r}\vdot\vb{r}'}{r})^n\\
=&\qty(1+\frac{\vb{r}\vdot\vb{r}'}{r^2})e^{ikr}\sum_{n=0}^\infty\frac{(-ik)^n}{n!}\frac{1}{r}\qty(\frac{\vb{r}\vdot\vb{r}'}{r})^n.
\end{split}
\end{align}
Due to estimate \eqref{ec:estimate_1} we can truncate the series to obtain
\begin{equation}
\frac{
e^{ik\|\vb{r}-\vb{r}'\|}}{\|\vb{r}-\vb{r}'\|}\approx\qty(1+\frac{\vb{r}\vdot\vb{r}'}{r^2})e^{ikr}\qty(\frac{1}{r}-ik\frac{\vb{r}\vdot\vb{r}'}{r^2}).
\end{equation}
Thus far, all of our estimates have been consistent up to $\mathcal{O}(\frac{\vb{r}\vdot\vb{r}'}{r^2})$. Therefore, we may estimate further
\begin{equation}
\frac{e^{ik\|\vb{r}-\vb{r}'\|}}{\|\vb{r}-\vb{r}'\|}\approx e^{ikr}\qty(\frac{1}{r}-ik\frac{\vb{r}\vdot\vb{r}'}{r^2})=\frac{e^{ikr}}{r}\qty(1-ik\frac{\vb{r}\vdot\vb{r}'}{r}).
\end{equation}
Pluggin this estimate in \eqref{ec:fourier_vector_potential} we obtain
\begin{equation}
\vb{A}(\vb{r},\omega)=\frac{\mu_0e^{ikr}}{4\pi r}\int \dd[3]{\vb{r}'}\qty(1-ik\frac{\vb{r}\vdot\vb{r}'}{r})\vb{J}(\vb{r}',\omega).
\end{equation}
Remembering \eqref{ec:integral_current} and the hint in the exam sheet we obtain
\begin{align}
\begin{split}\label{ec:magnetic_potential}
\vb{A}(\vb{r},\omega)=&\frac{\mu_0e^{ikr}}{4\pi r}\qty(-i\omega\vb{p}-\frac{ik}{r}\int\dd[3]{\vb{r}'}(\vb{r}\vdot\vb{r}')\vb{J}(\vb{r}',\omega))\\
=&-\frac{\mu_0e^{ikr}}{4\pi r}\qty(i\omega\vb{p}+\frac{ik}{2r}\int\dd[3]{\vb{r}'}(\vb{r}'\cross\vb{J}(\vb{r}',\omega))\cross\vb{r})\\
=&-\frac{i\mu_0e^{ikr}}{4\pi r}\qty(\omega\vb{p}+\frac{k}{r}\vb{m}\cross\vb{r}).
\end{split}
\end{align}

\item We can express the Fourier transform of the electric and magnetic field by
\begin{equation}
\vb{B}(\vb{r},\omega)=\int\dd{t}e^{i\omega t}\vb{B}(\vb{r},t)=\int\dd{t}e^{i\omega t}\curl{\vb{A}}(\vb{r},t).
\end{equation}
Given that $e^{i\omega t}$ has no spatial dependence and assuming that the right conditions are met to exchange differentiation and integration we obtain
\begin{equation}
\vb{B}(\vb{r},\omega)=\curl{\int\dd{t}e^{i\omega t}\vb{A}(\vb{r},t)}=\curl{\vb{A}}(\vb{r},\omega).
\end{equation}
Making use of the product rule 
\begin{align}
\begin{split}
\curl(f\vb{F})=&\epsilon_{ijk}\partial_j(fF_k)\vu{e}_i=\epsilon_{ijk}\partial_jfF_k\vu{e}_i+\epsilon_{ijk}f\partial_jF_k\vu{e}_i\\
=&\grad{f}\cross\vb{F}+f\curl{F}
\end{split}
\end{align}
we have through \eqref{ec:magnetic_potential}
\begin{align}
\begin{split}
\vb{B}(\vb{r},\omega)=&-\frac{i\mu_0}{4\pi}\grad(\frac{e^{ikr}}{r})\cross\qty(\omega\vb{p}+\frac{k}{r}\vb{m}\cross\vb{r})\\
&-\frac{i\mu_0e^{ikr}}{4\pi r}\curl(\omega\vb{p}+\frac{k}{r}\vb{m}\cross\vb{r}).
\end{split}
\end{align}
Given that $\vb{p}$ has no spatial dependence this can be reduced to
\begin{align}
\begin{split}
\vb{B}(\vb{r},\omega)=&-\frac{i\mu_0}{4\pi}\grad(\frac{e^{ikr}}{r})\cross\qty(\omega\vb{p}+\frac{k}{r}\vb{m}\cross\vb{r})\\
&-\frac{i\mu_0ke^{ikr}}{4\pi r}\curl(\vb{m}\cross\frac{\vb{r}}{r}).
\end{split}
\end{align}
Notice that given that $\vb{m}$ has no spatial dependence if $\vb{F}$ is a function of space
\begin{align}\label{ec:product_rule}
\begin{split}
\curl(\vb{m}\cross\vb{F})=&\epsilon_{ijk}\partial_j(\epsilon_{klm}m_lF_m)\vu{e}_i=\epsilon_{ijk}\epsilon_{klm}m_l\partial_jF_m\vu{e}_i\\
=&(\delta_{il}\delta_{jm}-\delta_{im}\delta_{jl})m_l\partial_jF_m\vu{e}_i\\
=&m_i\partial_jF_j\vu{e}_i-m_j\partial_jF_i\vu{e}_i=(\div{\vb{F}})\vb{m}-(\vb{m}\vdot\grad)\vb{F}.
\end{split}
\end{align}
Now we will begin to evaluate the necessary derivatives to obtain the final result. We note that in calculations we don't have to consider $\vb{r}=0$ since we are in the far zone. We have
\begin{equation}
\grad(\frac{e^{ikr}}{r})=\partial_i\qty(\frac{e^{ikr}}{r})\vu{e}_i=\frac{ike^{ikr}r\partial_ir-e^{ikr}\partial_ir}{r^2}\vu{e}_i
\end{equation}
Since
\begin{equation}
\partial_ir=\partial_i\sqrt{(x_1)^2+(x_2)^2+(x_3)^2}=\frac{2x_i}{2\sqrt{(x_1)^2+(x_2)^2+(x_3)^2}}=\frac{x_i}{r}
\end{equation}
we obtain
\begin{equation}\label{ec:other_gradient}
\grad(\frac{e^{ikr}}{r})=\frac{ike^{ikr}x_i-e^{ikr}x_i/r}{r^2}\vu{e}_i=e^{ikr}\qty(ik-\frac{1}{r})\frac{\vb{r}}{r^2}.
\end{equation}
On the other hand, 
\begin{equation}\label{ec:divergence}
\div(\frac{\vb{r}}{r})=\partial_i\qty(\frac{x_i}{r})=\frac{r-x_i\partial_ir}{r^2}=\frac{r-x_ix_i/r}{r^2}=\frac{r-r^2/r}{r^2}=0.
\end{equation}
Finally
\begin{equation}
\partial_i\qty(\frac{x_j}{r})=\frac{\delta_{ij}r-x_j\partial_ir}{r^2}=\frac{\delta_{ij}r-x_jx_i/r}{r^2}=\frac{\delta_{ij}}{r}-\frac{x_jx_i}{r^3}
\end{equation}
and thus
\begin{equation}\label{ec:weird_divergence}
(\vb{m}\vdot\grad)\frac{\vb{r}}{r}=m_i\partial_i\qty(\frac{x_j}{r})\vu{e}_j=m_i\qty(\frac{\delta_{ij}}{r}-\frac{x_jx_i}{r^2})\vu{e}_j=\frac{\vb{m}}{r}-(\vb{m}\vdot\vb{r})\frac{\vb{r}}{r^3}.
\end{equation}
Putting it all together we conclude
\begin{align}
\begin{split}
\vb{B}(\vb{r},\omega)=&\frac{i\mu_0e^{ikr}}{4\pi r^2}\qty(\frac{1}{r}-ik)\vb{r}\cross\qty(\omega\vb{p}+\frac{k}{r}\vb{m}\cross\vb{r})\\
&+\frac{i\mu_0ke^{ikr}}{4\pi r^2}\qty(\vb{m}-(\vb{m}\vdot\vb{r})\frac{\vb{r}}{r^2})\\
=&\frac{i\mu_0e^{ikr}}{4\pi r^2}\qty(\frac{1}{r}-ik)\vb{r}\cross\qty(\omega\vb{p}+\frac{k}{r}\vb{m}\cross\vb{r})\\
&+\frac{i\mu_0ke^{ikr}}{4\pi r^2}\qty(\vb{m}-(\vb{m}\vdot\vb{r})\frac{\vb{r}}{r^2})\\
=&\frac{i\mu_0e^{ikr}}{4\pi r^2}\qty(\frac{1}{r}-ik)\vb{r}\cross\qty(\omega\vb{p}+\frac{k}{r}\vb{m}\cross\vb{r})\\
&+\frac{i\mu_0ke^{ikr}}{4\pi r^2}\qty(\vb{m}-(\vb{m}\vdot\vb{r})\frac{\vb{r}}{r^2})\\
=&\frac{i\mu_0}{4\pi}\left(\omega\frac{e^{ikr}}{r}\qty(\frac{1}{r}-ik)\frac{\vb{r}}{r}\cross\vb{p}\right.\\
+&\left.k\frac{e^{ikr}}{r}\qty(\frac{1}{r}-ik)\frac{\vb{r}}{r}\cross\qty(\vb{m}\cross\frac{\vb{r}}{r})+k\frac{e^{ikr}}{r^2}\vb{m}\right.\\
&\left.-k\frac{e^{ikr}}{r^2}\qty(\vb{m}\vdot\frac{\vb{r}}{r})\frac{\vb{r}}{r}\right)\\
\approx&\frac{\mu_0k}{4\pi}\frac{e^{ikr}}{r}\qty(\omega\frac{\vb{r}}{r}\cross\vb{p}+k\frac{\vb{r}}{r}\cross\qty(\vb{m}\cross\frac{\vb{r}}{r}))
\end{split}
\end{align}
when only keeping the terms of leading order in $r^{-1}$ given that we are in the radiation zone and to remain consistent with the order of our previous approximations.

Recalling Maxwell's equation
\begin{equation}
\curl{\vb{B}}(\vb{r},t)=\mu_0\qty(\vb{J}(\vb{r},t)+\epsilon_0\pdv{\vb{E}}{t}(\vb{r},t))=\frac{1}{c^2}\pdv{\vb{E}}{t}(\vb{r},t)
\end{equation}
assuming that the fields are localized and thus, in the far zone, there are no sources and the current is null. Assuming the inverse Fourier transforms exist and that we can exchange orders of differentiation and integration
\begin{align}
\begin{split}
\frac{1}{2\pi}\int\dd{\omega}e^{-i\omega t}\curl{\vb{B}}(\vb{r},\omega)=&\curl\frac{1}{2\pi}\int\dd{\omega}e^{-i\omega t}\vb{B}(\vb{r},\omega)\\
=&\curl{\vb{B}}(\vb{r},t)=\frac{1}{c^2}\pdv{\vb{E}}{t}(\vb{r},t)\\
=&\frac{1}{c^2}\pdv{t}\frac{1}{2\pi}\int\dd{\omega}e^{-i\omega t}\vb{E}(\vb{r},\omega)\\
=&\frac{-i\omega}{2\pi c^2}\int\dd{\omega}e^{-i\omega t}\vb{E}(\vb{r},\omega).
\end{split}
\end{align}
Assuming the Fourier transform is unique we obtain
\begin{equation}
\vb{E}(\vb{r},\omega)=\frac{ic}{k}\curl{\vb{B}}(\vb{r},\omega).
\end{equation}
Recalling \eqref{ec:other_gradient}, \eqref{ec:product_rule}, \eqref{ec:divergence}, and \eqref{ec:weird_divergence} we may perform this derivative to obtain
\begin{align}
\begin{split}
\vb{E}(\vb{r},\omega)=&\frac{ic\mu_0}{4\pi}\grad(\frac{e^{ikr}}{r})\cross\qty(\omega\frac{\vb{r}}{r}\cross\vb{p}+k\frac{\vb{r}}{r}\cross\qty(\vb{m}\cross\frac{\vb{r}}{r}))\\
&+\frac{ic\mu_0}{4\pi}\frac{e^{ikr}}{r}\curl(\omega\frac{\vb{r}}{r}\cross\vb{p}+k\frac{\vb{r}}{r}\cross\qty(\vb{m}\cross\frac{\vb{r}}{r}))\\
\approx&-\frac{\mu_0\omega}{4\pi}\frac{e^{ikr}}{r}\frac{\vb{r}}{r}\cross\qty(\omega\frac{\vb{r}}{r}\cross\vb{p}+k\frac{\vb{r}}{r}\cross\qty(\vb{m}\cross\frac{\vb{r}}{r}))\\
=&\frac{\mu_0\omega}{4\pi}\frac{e^{ikr}}{r}\qty(\omega\frac{\vb{r}}{r}\cross\vb{p}+k\frac{\vb{r}}{r}\cross\qty(\vb{m}\cross\frac{\vb{r}}{r}))\cross\frac{\vb{r}}{r}\\
\end{split}
\end{align}
up to leading order in $r^{-1}$. Notice that it is not even worth to calculate the curl once the product rule is done since it will be of higher order in $r^{-1}$ as we learned from the calculation of the magnetic field. Moreover, we obtain that $\vb{E}$, $\vb{B}$, and the direction of propagation (which in the far zone is expected to be $\vb{r}/r$ if the sources are local) form an orthogonal proper system. We expected this from the elementary theory of radiation. Using the triple product
\begin{align}
\begin{split}
\vb{A}\cross(\vb{B}\cross\vb{C})=&\epsilon_{ijk}A_j\epsilon_{klm}B_lC_m\vu{e}_i=(\delta_{il}\delta_{jm}-\delta_{im}\delta{jl})A_jB_lC_m\vu{e}_i\\
=&A_jB_iC_j\vu{e}_i-A_jB_jC_i\vu{e}_i=(\vb{A}\vdot\vb{C})\vb{B}-(\vb{A}\vdot\vb{B})\vb{C}
\end{split}
\end{align} 
we have
\begin{align}
\begin{split}
(\vb{A}\cross(\vb{B}\cross\vb{A}))\cross\vb{A}=(A^2\vb{B}-(\vb{A}\vdot\vb{B})\vb{A})\cross\vb{A}=A^2\vb{B}\cross\vb{A}.
\end{split}
\end{align}
Therefore, by setting $\vb{A}=\vb{r}/r$, $\vb{B}=\vb{m}$, and noticing that $A^2=1$, we have
\begin{equation}
\vb{E}(\vb{r},\omega)=\frac{\mu_0\omega}{4\pi}\frac{e^{ikr}}{r}\qty(\omega\frac{\vb{r}}{r}\cross\vb{p}+k\vb{m})\cross\frac{\vb{r}}{r}.
\end{equation}

\end{enumerate}

\end{document}
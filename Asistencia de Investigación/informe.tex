\documentclass{article}

\usepackage[utf8]{inputenc}
\usepackage{amsmath}
\usepackage{enumerate}
\usepackage{amssymb}
\usepackage{amsthm}
\usepackage{physics}
\usepackage{siunitx}

\newtheorem{definition}{Definition}
\newtheorem{example}{Example}
\newtheorem{theorem}{Theorem}

\DeclareMathOperator{\A}{\mathcal{A}}
\DeclareMathOperator{\Hil}{\mathcal{H}}
\DeclareMathOperator{\BH}{\mathcal{B}(\Hil)}
\DeclareMathOperator{\id}{id}
\DeclareMathOperator{\C}{\mathbb{C}}
\DeclareMathOperator{\Hom}{Hom}

\title{Formato de Presentación de Informes para emisión de paz y salvos de Asistencias Intersemestrales (junio-julio 2018)
\\
Fecha de entrega: 15 de agosto 2018 (fecha de envío al profesor asesor)}

\author{}

\date{}

\begin{document}

\maketitle

\section{Nombre del estudiante y código:}

Iván Mauricio Burbano Aldana\\
201423205
$\si{angstrom}$

\section{Programa de posgrado al que pertenece}

Maestría en Ciencias-Física

\section{Nombre del profesor asesor}

Andrés Fernando Reyes Lega

\section{Título de la propuesta}

Ambigüedades cuánticas en entropías y teoría modular

\section{Objetivos propuestos}

En este trabajo se pretende seguir trabajando en un artículo junto con el profesor Aiyalam Parameswaran Balachandran, de la Universidad de Syracuse, la candidata doctoral Souad Maria Tabban Sabbagh y el profesor Andrés Fernando Reyes Lega de la Universidad de los Andes. Este artículo estudia el papel del conmutante $\mathcal{A}'$ del álgebra de observables $\mathcal{A}$ en la caracterización de las distintas descomposiciones en irreducibles del espacio GNS $\mathcal{H}_\omega$. Se ha observado que el conmutante mantiene invariante el valor esperado de los observables y por lo tanto tiene interpretación de álgebra gauge. Esto permite entender esta ambigüedad en la entropía en términos de una simetría gauge emergente. El estudio de esta álgebra gauge se puede hacer mediante la teoría modular de Tomita-Takesaki. Esto provee una manera de relacionar la teoría modular con esta anomalía cuántica. Balachandran ha propuesto que tal relación no solamente se encuentra en las ambigüedades de entropía y puede ser extendida al estudio general de la anomalías cuánticas.

Hasta el momento se ha logrado interpretar esta ambigüedad en términos de la teoría modular en el caso de estados normales sobre álgebras de operadores sobre espacios de Hilbert de dimensión finita. Se espera que mediante el estudio de la molécula de etileno\cite{Balachandran2013a} se pueda extender este resultado a los sistemas generados mediante la cuantización de espacios de configuración homogéneos. De esta manera podemos obtener más ejemplos que nos ayuden a entender el papel de la teoría modular en la anomalías cuánticas. De forma simultanea se va a estudiar el método de techos convexos para obtener una interpretación física más clara de esta ambigüedad\cite{Uhlmann2010}. En efecto, es necesario extender la idea de entropía de enredamiento a sistemas compuestos cuyo estado no es puro. Se ha observado que en estos sistemas la entropía es anómala para el operador densidad total y su reducción a la componente gauge. Esto indica que mediante la teoría modular podemos caracterizar el conjunto de estados en el sistema completo posibles a partir del conocimiento del estado en uno de los subsistemas.

\section{Resultados encontrados}

Se realizó un viaje a la escuela y congreso Noncommutative Geometry and Poisson Geometry Around Groupoids en Villa de Leyva. Durante este se hicieron dos avances importantes. Por una parte, mediante la discusión con el profesor Bas Janssens de TU Delft se encontró una descripción de los operadores modulares en álgebras matriciales de dimensión finita sin necesidad de bases. Una versión de esta descripción se puede ver en unas notas (no terminadas) del profesor\cite{Janssens2013}. Por otra parte, el profesor Eli Hawkins de University of York ofreció unas conferencias sobre cuantización geométrica y grupoides. En esta se trataron temas que son de relevancia para la cuantización del espacio de configuración de la molécula de etileno. Aunque el formalismo general de cuantización es bastante técnico, los temas importantes para su entendimiento quedaron claros y hay una idea de cómo proseguir.  En particular, es necesario continuar con el estudio de $C^*$-álgebras de grupos, polarizaciones, medias densidades y grupoides simplécticos.

Como paso inicial para el tratamiento de la molécula de etileno, se decidió realizar el estudio del problema de una partícula en un círculo. Aunque este es un espacio homogéneo, el grupo de isotropía es abeliano y por lo tanto no se espera ver la misma anomalía en entropía de la molécula de etileno. Sin embargo, este ejemplo si presenta otros tipos de anomalías\cite{Balachandran2011}. Además, encontramos que la cuantización de este sistema es una subalgebra $\mathcal{A}_{S^1}$ de la $C^*$-álgebra de Weyl de $T^*\mathbb{R}$. Esta $C^*$-álgebra es bien conocida\cite{Moretti2013}. Debido a esto, hemos podido obtener resultados preliminares sobre su teoría de representaciones. En particular, considere el Hamiltoniano 
\begin{equation}
H_\theta=-\frac{1}{2}\dv[2]\phi
\end{equation} 
con dominio $D_\theta=\{\psi\in L^2([0,2\pi])|\psi(2\pi)=e^{i\theta}\psi(0)\}$. Hemos visto que la construcción GNS de $\mathcal{A}_{S^1}$ con respecto a los estados base de este Hamiltoniano llevan directamente a su dominio. Además, se tiene que $D_\theta$ no es invariante bajo el operador de paridad $P\psi(\phi)=\psi(\phi-2\pi)$. Sin embargo, introduciendo estados mixtos se restaura esta simetría. Hemos podido demostrar que estos son los únicos estados bases invariantes bajo paridad.

Nuestra segunda aproximación al problema de etileno fue mediante el estudio de la teoría de representaciones del álgebra de grupo. En este nos planteamos el problema de, dado un grupo finito $G$ y una representación irreducible $\pi:G\rightarrow\mathcal{U}(V)$, hallar un operador densidad $\rho$ actuando en $L^2(G)$ tal que la representación GNS del estado inducido $\omega=\tr(\rho*\cdot)$ en el álgebra de grupo $\mathcal{A}(G)$ sea isomorfa a $\Hom V$. Es bien conocido que si $\{\pi^\alpha:G\rightarrow V^\alpha|\alpha\in\hat{G}\}$ son todas las representaciones irreducibles de $G$ salvo equivalencia entonces 
\begin{equation}
\mathcal{A}(G)\cong\bigoplus_{\alpha\in\hat{G}}\Hom V^\alpha.
\end{equation}
Se pudo ver que si se toma $\rho\in\Hom V^\alpha$ entonces $\mathcal{H}_\omega=\Hom V^\alpha$. Además, el ejemplo explicito para $G=S_3$ se llevó a cabo.

En el problema de interpretación para la ambigüedad de la entropía surgió la idea de hacer uso de entropías relativas. La propuesta fue dada una vez más por el profesor Janssens. El argumentaba que la ambigüedad en la entropía que hemos encontrado se debe a un cambio en la noción de \textit{estado uniforme}. Usualmente se toma como estado uniforme al tracial. En mecánica clásica esto corresponde a tomar la medida de Liouville del espacio de fase como noción \textit{natural} de tamaño. Sin embargo, esto pierde sentido en sistemas infinitos. Es por esto necesaria la introducción de entropías de un estado con respecto a otro. Siguiendo esta idea, se intentó explicar la ambigüedad en terminos de una escogencia ambigüa en el estado de referencia de los cálculos en entropía. Este es un trabajo en proceso.

Finalmente, asistí a la escuela Journeys into Theoretical Physics. Con esta participación se espera haber fortalecido las relaciones entre IFT-Perimeter-Saifr y la Universidad de los Andes.

\vspace{1.5cm}
Iván Mauricio Burbano Aldana\\

Nombre y firma del estudiante

\vspace{1.5cm}
Vo.Bo. del profesor asesor

\bibliography{../Mendeley/library}
\bibliographystyle{ieeetr}

\end{document}

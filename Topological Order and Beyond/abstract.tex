\documentclass{article}

\usepackage[utf8]{inputenc}
\usepackage[spanish, english]{babel}

\title{Entropy Ambiguities and Modular Theory}
\author{Iván Mauricio Burbano Aldana}

\begin{document}

\maketitle

\begin{abstract}

The lack of an axiomatic framework for Quantum Field Theory has brought forward attention to the algebraic formulation of physical theories\cite{Haag1992}. The  power of such a description lies on the clearness of the physical and mathematical interpretation of the objects involved. However, by losing explicit reference to a Hilbert space in the description of a system, there is no clear way to define the entropy of a state. By using the GNS representation, one can circumvent this problem and calculate the von Neumann entropy of a density operator associated to the state. This method introduces an ambiguity due to the density operator depending on a decomposition into irreducible representations of the GNS space\cite{Balachandran2013b}. In a work in progress with professor Andrés Fernando Reyes and the Ph.D. candidate Souad Marías Tabban, we show that this ambiguity can be studied using the Tomita-Takesaki modular theory. Moreover, we provide clues towards a physical interpretation by mapping this problem into that of finding those density operators on a composite system which restrict to a given state on one of its constituents.  

\end{abstract}

\begin{otherlanguage}{spanish}

\begin{abstract}

La falta de un marco axiomático para la Teoría Cuántica de Campos ha aumentado la relevancia de las formulación algebraica de las teorías físicas\cite{Haag1992}. Esta descripción es prometedora debido a la transparencia de la interpretación física y matemática de los objetos involucrados. Sin embargo, al dejar de lado un espacio de Hilbert en la descripción del sistema, no hay una forma clara de definir la entropía de un estado. Uno puede evitar este problema utilizando la construcción GNS y calculando la entropía de von Neumann de un operador densidad asociado al estado. Este método introduce una ambigüedad debido a que el operador densidad depende de una descomposición en irreducibles del espacio GNS\cite{Balachandran2013b}. En un trabajo en progreso con el profesor Andrés Fernando Reyes y la candidata doctoral Souad María Tabban, mostramos que esta ambigüedad se puede estudiar mediante la teoría modular de Tomita-Takesaki. Además, damos pistas hacia una interpretación física relacionando este problema al de encontrar aquellos operadores densidad en un sistema compuesto que se restringen a un estado dado en una de sus componentes.

\end{abstract}

\end{otherlanguage}

\nocite{*}

\bibliography{references}
\bibliographystyle{ieeetr}

\end{document}
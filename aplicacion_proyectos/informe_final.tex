\documentclass{article}

\usepackage[utf8]{inputenc}
\usepackage[spanish]{babel}
\PassOptionsToPackage{hyphens}{url}\usepackage{hyperref}

\title{Informe Final: Ambigüedades Cuánticas en Entropía y Teoría Modular}
\author{Iván Mauricio Burbano Aldana \\ Avalado por Andrés Fernando Reyes Lega \\ Proyecto: INV-2018-34-1295 }

\begin{document}

\maketitle

\section{Productos}

\subsection{Participación en Evento Científico}

Si bien inicialmente se propuso presentar los resultados durante el \href{https://2018fismatcl.wordpress.com/}{Mes de la Física-Matemática en Chile}, no fue posible conseguir el espacio necesario. Aún así, se participó en la conferencia \href{http://eventos.cmm.uchile.cl/rps-patagonia2018/}{Random Physical Systems}, como se puede verificar en la lista de participantes: \url{http://eventos.cmm.uchile.cl/rps-patagonia2018/participants/}. En esta se crearon lazos importantes con la comunidad física-matemática chilena que esperamos sean útiles en futuras investigaciones. Por otro lado, el trabajo realizado se presentó en la \href{https://www.ictp-saifr.org/4th-joint-dutch-brazil-school-on-theoretical-physics/}{4th Dutch-Brazil School on Theoretical Physics}, bajo el título Emerging Gauge Symmetries and Quantum Operations. La participación se puede verificar en la lista de participantes: \url{https://www.ictp-saifr.org/wp-content/uploads/2019/02/Lista-de-Participantes-4th-Joint-Dutch-Brazil-School.pdf}. Adjunto también se encuentra el certificado de participación donde se verifica la presentación del trabajo desarrollado.

\subsection{Artículo}

\end{document}
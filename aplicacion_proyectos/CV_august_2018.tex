\documentclass[line,margin]{res}

\usepackage[utf8]{inputenc}
\usepackage{hyperref}

\begin{document}

\name{Iván Mauricio Burbano Aldana}
\address{ivanmbur@gmail.com\\ Cr. 19 63 27, Bogotá, Colombia\\ (+57) 316 782 1110}

\begin{resume}

\section{About me}

Born on December 26, 1996 in Bucaramanga, Colombia, I've always been characterized by my strong passions. My first discoveries on symmetry through music laid the basis for my passion to understand what surrounds us. Physics has offered me a path through which I can explore the world making use of my creativity. At this point in my life I've decided to take advantage of every chance I get along the way to learn new things, meet new people, and become a better person. I am certain that the success of this enterprise will be guaranteed by the discipline and commitment I give to my work. 

\section{Special Skills}

{\sl Theoretical Physics}: Geometric, algebraic, and topological methods in physics.

{\sl Mathematics}: Algebra, Topology, Analysis, Measure Theory, and Algebraic Topology.

{\sl Programming}: Unix/Linux, Java, Python, C, IRIS, and DS9.

{\sl Experimental Physics}: Infrared sensors and electrochemically exfoliated graphene. 

{\sl Languages}: Spanish and English ({\sl TOEFL} 113/120). Three semesters of german studies.

\section{Education}

{\sl Physics Masters\\}
Universidad de los Andes, Bogotá, Colombia \\
In progress

{\sl Physics Major\\}
Universidad de los Andes, Bogotá, Colombia \\
Cum Laude \\
Mathematics Minor \\
Thesis: KMS States and Tomita-Takesaki Theory \\
March 2018 \\
Average: 4.72/5

{\sl Highschool degree: Academic Bachelor} \\
San Carlos School, Bogotá, Colombia \\
June 2014

\section{Awards}

\begin{itemize}

\item {\sl Cum Laude Undergraduate Degree} for graduating with grades amongst the top 3\% of the past five years of the science faculty at the Universidad de los Andes.  

\item {\sl Semiannual Excellence Prize} for the best grades on the first semester of 2017 of the Physics department at the Universidad de los Andes.

\item {\sl Ramón de Zubiría Prize} for the best global grades up to the first semester of 2016 of the Physics department at the Universidad de los Andes.

\item {\sl Semiannual Excellence Prize} for the best grades on the second semester of 2015 of the Physics department at the Universidad de los Andes.

\item {\sl Honorable mention} in the 45th International Physics Olympiad (Kazakhstan 2014).

\item {\sl Third place} in the Colombian Physics Olympiad of 2013.

\end{itemize}

\section{Experience}

{\sl SURF California Institute of Technology}: I worked under the supervision of Dr. Roger Smith and Dr. Andrés Plazas during the summer of 2017. I aided the investigation, characterization, and correction of image centroid motions at the Precision Projector Laboratory in the Jet Propulsion Laboratory (NASA). The detectors investigated will be used for the study of weak gravitational lenses and dark matter in missions such as WFIRST and Euclid.

{\sl Teacher at the Universidad de los Andes}:
\begin{itemize}

\item Physics 2 recitation in the second semester of 2018.

\item Experimental Physics 1 in the second semester of 2018.

\item Physics 1 recitation in the first semester of 2018.

\item Basic Physics 1 recitation in the first semester of 2018.

\end{itemize}

{\sl Teaching assistant at the Universidad de los Andes}: 
\begin{itemize}

\item Lineal Algebra 2 with professor César Galindo in the second semester of 2016.

\item Physics clinic in the second semester of 2016.

\item Linear Algebra (Honors) with professor Sergio Adarve in the first semester of 2016.

\end{itemize}

{\sl Nanomaterials Laboratory}: during the second semester of 2016 I researched the actuating properties of electrochemically exfoliated graphene under the supervision of professor Yenny Hernández as part of the Intermediate Lab course.
  
{\sl Tutor}: I have helped students of Physics 1, Physics 2, Integral Calculus, Vector Calculus, Linear Algebra 2, and Mathematical Methods.

\section{Seminars}

{\sl Noncommutative Geometry and Poisson Geometry Around Groupoids School and Conference:} Applications of Tomita-Takesaki Theory to Quantum Physics I.\\
Villa de Leyva, Colombia
July 2018

{\sl Quantum Optics Seminar:} Quantum Logic and the Orthocomplemented Lattice of Propositions: A logic based approach to Bell's inequalities.\\
Universidad de los Andes, Bogotá, Colombia\\
June 2018

{\sl Topological Order and Beyond:} KMS States and Tomita-Takesaki Theory.\\
Universidad de los Andes, Bogotá, Colombia\\
June 2018\\
\url{https://matematicas.uniandes.edu.co/~cursillo_gr/escuela2018/abstracts.php#abstract_burbano}

{\sl Quantum Field Theory and Mathematical Physics Seminar:} Algebraic Formulation of Quantum Physics.\\
Universidad de los Andes, Bogotá, Colombia\\
February 2018

{\sl La Cicuta Magazine 7th edition release}: On the Communication and Censorship of Science.\\
Universidad de los Andes, Bogotá, Colombia\\
September 2017

{\sl Quantum Field Theory and Mathematical Physics Seminar:} Orthocomplemented Quantum Lattices of Propositions.\\
Universidad de los Andes, Bogotá, Colombia\\
April 2017
  
  
\section{Seminars Attended}

{\sl Journeys into Theoretical Physics:}\\
IFT-Perimeter-Saifr\\
São Paulo, Brasil\\
July 2018

{\sl Noncommutative Geometry and Poisson Geometry Around Groupoids School and Conference:}\\
Villa de Leyva, Colombia
July 2018

{\sl Topological Order and Beyond:}\\
Universidad de los Andes, Bogotá, Colombia\\
June 2018

{\sl Dynamics of Quantum Systems Outside of Equilibrium} \\
Universidad de los Andes, Bogotá, Colombia\\
December 2017

{\sl URDiplomats workshop} \\
Universidad del Rosario, Bogotá, Colombia \\
April 2014  
 
\section{Extracurricular Activities}

\begin{itemize}

\item Tutoring at the Iglesia de Nuestra Señora de las Aguas for the state exam. This was targeted at low income teenagers of La Candelaria neighborhood. 

\item Guitar classes targeted at low income teenagers. This was part of my social service at high school.

\item Elected president of the student council at Colegio San Carlos during my senior year.

\item I teamed up with the United Nations Information Center to design the first UN model at a citywide scale SIMONU 2013. I also acted as the Secretary General of the UN model of my school SACMUN X and performed as president and delegate of various other models in Bogotá.

\end{itemize}  
  
\section{References}  

\begin{itemize}

\item Prof. Andrés Fernando Reyes Lega: anreyes@uniandes.edu.co

\item Dr. Roger Smith: rsmith@astro.caltech.edu 

\item Dr. Andrés Plazas: andres.a.plazas.malagon@jpl.nasa.gov

\item Prof. Carlos Andrés Flórez Bustos: ca.florez@uniandes.edu.co

\item Prof. Sergio Adarve: sadarve@uniandes.edu.co

\item Prof. César Neyit Galindo Martínez: cn.galindo1116@uniandes.edu.co

\end{itemize}
  
\end{resume}
\end{document}

\documentclass{article}

\usepackage[utf8]{inputenc}
\usepackage[english, spanish]{babel}
\usepackage{authblk}
\usepackage{hyperref}

\title{Emergent Time from Quantum Variability\\{\small (Tiempo emergente de la variabilidad cuántica)}}

\usepackage{amssymb}
\usepackage{amsmath}

\DeclareMathOperator{\Out}{Out}

\author{Iván Mauricio Burbano Aldana\thanks{\href{mailto:im.burbano10@uniandes.edu.co}{im.burbano10@uniandes.edu.co}}\\
Code: 201423205
\\Advised by: Dr. Andrés Fernando Reyes Lega\thanks{\href{mailto:anreyes@uniandes.edu.co}{anreyes@uniandes.edu.co}}}
\affil{Departamento de F\'{i}sica, Universidad de los Andes,  A.A. 4976-12340, Bogot\'a, Colombia}

\begin{document}

\selectlanguage{english}

\maketitle

\begin{abstract}

Operator algebras, as general frameworks for the description of the observables of physical systems, have been an important tool towards rigorous investigations in information theory, statistical physics, and quantum field theory\cite{Ohya1993, Bratteli1987, Bratteli1997, Connes2008}. In this project we will familiarize with these techniques and explore Connes' proposal for the emergence of time as a quantum phenomenon\cite{Connes2015}. By building from \cite{Burbano2017}, we will begin through the study of a generalization of the Radon-Nikod\'ym theorem to von Neumann algebras\cite{Connes1973}. This central result will yield a canonical dynamical mapping $\mathbb{R}\rightarrow\Out(\mathcal{M})$ due to the noncommutativity of the von Neumann algebra $\mathcal{M}$. For type III von Neumann algebras, which are intimately related to the physics of systems with an infinite number of degrees of freedom\cite{Yngvason2004}, this prescription provides a class of dynamical evolutions which differ only locally from each other. It is in this sense that $\mathcal{M}$ should be regarded by itself as a dynamical object. After understanding the theoretical and mathematical details of this construction, we will aim to present in great detail physically meaningful examples. We hope that this exercise will shed light into the physical meaning of Connes' mathematical proposal and guide research towards more concrete formulations.

\end{abstract}

\selectlanguage{spanish}

\begin{abstract}

Las álgebras de operadores, como marcos generales para la descripción de observables en sistemas físicos, han sido una herramienta importante para investigaciones rigurosas en teoría de la información, física estadística y teoría cuántica de campos\cite{Ohya1993, Bratteli1987, Bratteli1997, Connes2008}.  En este proyecto vamos a familiarizarnos con estas técnicas y exploraremos la propuesta de Connes sobre la emergencia del tiempo como un fenómeno cuántico\cite{Connes2015}. Partiendo de \cite{Burbano2017}, empezaremos estudiando una generalización del teorema de Radon-N\`ykodym para álgebras de von Neumann\cite{Connes1973}. Este resultado central presentará un mapa dinámico canónico $\mathbb{R}\rightarrow\Out(\mathcal{M})$ debido a la no-conmutatividad del álgebra de von Neumann $\mathcal{M}$. Para álgebras de von Neumann tipo III, las cuales están intimamente relacionadas con la física de sistemas con un número infinito de grados de libertad\cite{Yngvason2004}, esta prescripción provee una clase de evoluciones dinámicas que solo difieren localmente entre si. Es en este sentido que $\mathcal{M}$ debe de ser entendido como un objeto dinámico por si mismo. Después de entender los detalles teóricos y matemáticos de esta construcción, pretenderemos mostrar en detalle ejemplos de importancia física. Esperamos que este ejercicio esclarezca el significado físico de la propuesta matemática de Connes y guíe la investigación hacía formulaciones más concretas.

\end{abstract}

\selectlanguage{english}

\section*{Recommended Jury}

\subsection*{Universidad de los Andes}

\begin{enumerate}

\item Alexander Cardona Guio

Associate Professor of the Mathematics Department

\href{mailto:acardona@uniandes.edu.co}{acardona@uniandes.edu.co}

CV: \url{https://pentagono.uniandes.edu.co/~acardona/ACcurriculumvitae.pdf}

\item Nelson Javier Buitrago Aza

Postdoctoral Researcher of the Physics Department

\href{mailto:nj.buitragoa@uniandes.edu.co}{nj.buitragoa@uniandes.edu.co}

\item Pedro Bargueño de Retes

Faculty Professor of the Physics Department

\href{mailto:p.bargueno@uniandes.edu.co}{p.bargueno@uniandes.edu.co}

CV: \url{http://scienti.colciencias.gov.co:8081/cvlac/visualizador/generarCurriculoCv.do?cod_rh=0001572334}

\end{enumerate}

\subsection*{Other}

\begin{enumerate}

\item Aiyalam Parameswaran Balachandran

Professor Emeritus at Syracuse University

\href{mailto:balachandran38@gmail.com}{balachandran38@gmail.com}

\item Jorge Andrés Plazas Vargas

Assistant Professor at Universidad Javeriana

\href{mailto:jorge.plazas@javeriana.edu.co}
{jorge.plazas@javeriana.edu.co}

CV: \url{http://scienti.colciencias.gov.co:8081/cvlac/visualizador/generarCurriculoCv.do?cod_rh=0001544588}

\item Leonardo Arturo Cano García 

Professor at Univesidad Nacional

\href{mailto:lcanog@unal.edu.co}{lcanog@unal.edu.co}

CV: \url{http://scienti.colciencias.gov.co:8081/cvlac/visualizador/generarCurriculoCv.do?cod_rh=0001450991}

\end{enumerate}

\section*{Signatures}

\vspace{2cm}

Iván Mauricio Burbano Aldana \hspace{2cm} Andrés Fernando Reyes Lega 

\bibliography{../Mendeley/library}
\bibliographystyle{unsrturl}

\end{document}
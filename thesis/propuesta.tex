\documentclass{article}

\usepackage[utf8]{inputenc}
\usepackage[english, spanish]{babel}
\usepackage{authblk}
\usepackage{hyperref}
\usepackage{amssymb}
\usepackage{amsmath}
\usepackage{physics}
\usepackage{csquotes}

\DeclareMathOperator{\Out}{Out}
\DeclareMathOperator{\M}{\mathcal{M}}
\DeclareMathOperator{\A}{\mathcal{A}}
\DeclareMathOperator{\h}{\mathcal{H}}
\DeclareMathOperator{\BH}{\mathcal{B}(\h)}
\DeclareMathOperator{\Aut}{Aut}

\newtheorem{definition}{Definition}[section]
\newtheorem{theorem}{Theorem}[section]

\title{Emergent Time from Quantum Variability\\{\small (Tiempo emergente de la variabilidad cuántica)}}

\author{Iván Mauricio Burbano Aldana\thanks{\href{mailto:im.burbano10@uniandes.edu.co}{im.burbano10@uniandes.edu.co}}\\
Code: 201423205
\\Advised by: Dr. Andrés Fernando Reyes Lega\thanks{\href{mailto:anreyes@uniandes.edu.co}{anreyes@uniandes.edu.co}}}
\affil{Departamento de F\'{i}sica, Universidad de los Andes,  A.A. 4976-12340, Bogot\'a, Colombia}

\begin{document}

\selectlanguage{english}

\maketitle

\begin{abstract}

Operator algebras, as general frameworks for the description of the observables of physical systems, have been an important tool towards rigorous investigations in information theory, statistical physics, and quantum field theory\cite{Ohya1993, Bratteli1987, Bratteli1997, Connes2008}. In this project we will familiarize with these techniques and explore Connes' proposal for the emergence of time as a quantum phenomenon\cite{Connes2015}. By building from \cite{Burbano2017}, we will begin through the study of a generalization of the Radon-Nikod\'ym theorem to von Neumann algebras\cite{Connes1973}. This central result will yield a canonical dynamical mapping $\mathbb{R}\rightarrow\Out(\mathcal{M})$ due to the noncommutativity of the von Neumann algebra $\mathcal{M}$. For type III von Neumann algebras, which are intimately related to the physics of systems with an infinite number of degrees of freedom\cite{Yngvason2004}, this prescription provides a class of dynamical evolutions which differ only locally from each other. It is in this sense that $\mathcal{M}$ should be regarded by itself as a dynamical object. After understanding the theoretical and mathematical details of this construction, we will aim to present in great detail physically meaningful examples. We hope that this exercise will shed light into the physical meaning of Connes' mathematical proposal and guide research towards more concrete formulations.

\end{abstract}

\section{State of the Art}

Undoubtably, the flow of time is characteristic in our daily interactions with the universe. This is manifest in theories which are not generally covariant and are thus described by observables, states, and time flows (usually generated by preferred Hamiltonians). We show how these appear in the setting of operator algebras in section \ref{sec:theory} below. It is however, not clear how this last element of a physical theory appears in the setting of general covariant theories. Based on the fact that states induce dynamics on the algebra of observables, and these are precisely those on which the states are perceived as in thermal equilibrium, the hypothesis of thermal time was formulated\cite{Connes1994a}. It states
\begin{displayquote}
The physical time depends on the state. When the system is in a state $\omega$, the physical time is given by the modular group $\alpha_{t/\beta}$ of $\omega$.
\end{displayquote}
Let us list some of the facts that support this hypothesis:
\begin{enumerate}
\item One recovers Schrödinger's equation from the Gibbs postulate for systems with a finite number of degrees of freedom. 
\item The relationship between the Gibbs postulate and the Hamilton equations is recovered for classical systems.
\item The Unruh temperature is recovered by comparing the physical time associated with an accelerated observer moving in vacuum and the modular group of the vacuum state on the Rindler wedge\cite{Bisognano1975, Bisognano1976}. The same is true for the Hawking temperature.
\item The thermodynamical time obtained from the cosmic background radiation coincides with the conventional Friedman-Robertson-Walker time\cite{Rovelli1993a}.
\end{enumerate}
While this proposal is quite dramatic in its outreach, and thus, very much controversial, it does seem to provide a setting for the study of the statistical nature of spacetime and insight towards the appeareance of time in quantum gravity\cite{Rovelli2009, Paetz2002, Martinetti2013}.

At another level of abstraction, given an algebra of observables, the modular groups are all related by inner automorphisms\cite{Connes1973}. This provides a mapping which assigns to every time an outer automorphism independently of any choice of state. This mapping has found important applications of mathematical interest such as the classification of type III von Neumann algebras. Moreover, Connes has proposed that this mapping is of fundamental physical importance and can be interpreted as the emergence of time through quantum variability\cite{Connes2015}. He argues that, much like noncommutativity is essential for the coexistence of variables which achieve a continuous or discrete set of values (through the spectra of operators), it is the noncommutativity of this variables which gives rise to time. Indeed, it is precisely in this setting that this canonical mapping is not trivial. This proposal however, does seem to be at a very early stage in the physical community, being dominated by blogposts and speculation rather than explicit calculations on physical systems\cite{Connes2007, Vystavillubosmotlv2016}.

\section{Objectives}\label{sec:objectives}

The main objective of this thesis is to explore the possible applications to physics of the canonical dynamics $\mathbb{R}\rightarrow\Out\M$ every von Neumann algebra $\M$. We expect to do this through the concrete construction of physically relevant examples which we may use to probe Connes' proposal. In order to achieve this we have set the following intermediate objectives:
\begin{enumerate}
\item Review the mathematics of the theory of operator algebras concerning KMS states, the Tomita-Takesaki theory, modular groups, and the Connes cocycle.
\item Explore the thermal time hypothesis through previously constructed modular groups\cite{Connes1994a, Bisognano1975, Bisognano1976, Rovelli1993a, Borchers1999}.
\item Give explicit calculations of the canonical dynamics $\mathbb{R}\rightarrow\Out\M$ for different algebras of observables.
\item Interpret these results in relation to the emergence of time through noncommutative structures.
\end{enumerate}

\section{Theoretical Framework}\label{sec:theory}

In quantum mechanics observables are identified with operators on a separable Hilbert space $\h$. The spectrum of such an operator is then interpreted as the possible set of values the observation can yield. Therefore, the operators representing observables must be selfadjoint. Given that real observations are represented by concrete measurement devices whose range of outcomes is always bounded, we conclude that the observables must be in the bounded operators $\BH$\footnote{One can alternatively claim that, in light of the spectral theorem\cite{Hall2013}, every unbounded operator can be understood in terms of its spectral projections. These are in particular bounded operators. This point of view corresponds to adopting the position that every measurement can be ultimately broken down into a (possibly uncountably infinite) set of propositions.}. The following mathematical structures implement these requirements and are thus candidate structures for the algebra of observables of physical systems.    
\begin{definition}
	A $C^*$-algebra $\A$ is a Banach *-algebra which satisfies the $C^*$-condition, that is, for every $a\in\mathcal{A}$ we have
	\begin{equation}
		\|a^*a\|=\|a\|^2.
	\end{equation}
	A von Neumann algebra ($W^*$-algebra for short) $\M$ is a $C^*$-algebra for which there exists a predual, that is, a Banach space $\M_*$ such that $\M=(\M_*)^*$ as Banach spaces.
\end{definition}
On the other hand, physical states, which constitute the physical information required to evaluate the expected value of observables, can aslo be defined in this setting.
\begin{definition}
	Let $\A$ be a $C^*$-algebra. A state on $\A$ is a normalized positive linear functional $\omega:\A\rightarrow\mathbb{C}$. It is said to be faithful if $\omega(a^*a)=0$ implies $a=0$ for all $a\in\A$. If $\M$ is a von Neumann algebra, a state $\omega$ on $\M$ is said to be normal if $\omega\in\M_*$ under the canonical embedding $\M_*\subseteq\M^*$.
\end{definition}
Note that in the context of von Neumann algebras and through the restriction to normal states, the structure of observables and states is complementary. This is evidence of the fact that each of them is meaningless without the other and, in fact, determines it. Finally, we can also consider dynamical evolutions in this framework. For definiteness we will work in the Heisenberg picture.
\begin{definition}
	A dynamical evolution on a $C^*$-algebra $\A$ is a strongly continuous representation $\alpha:\mathbb{R}\rightarrow\Aut\A$. A dynamical evolution on a $W^*$-algebra $\M$ is a weakly continuous representation $\alpha:\mathbb{R}\rightarrow\Aut\M$. For both of this the notation $\alpha_t:=\alpha(t)$ is common.
\end{definition} 
The relationship between this framework and the one usually used in classical and quantum formulations is given by the Gelfand-Naimark theorem and its commutative version\cite{Gelfand1943}, Sakai's theorem on the relation between abstract and concrete $W^*$-algebras\cite{Sakai1956}, its commutative analogue, the Riesz representation theorem\cite{Riesz1909, Kukutani1941}, the Radon-Nikod\'ym theorem\cite{Otton1930}, and  Gleason's theorem\cite{Gleason1957}.   

In this setting thermal equilibrium is characterized by the Kubo-Martin-Schwinger condition\cite{Haag1967, Kubo1957, Martin1959}. We will use the characterization given by \cite{Duvenhage1999}.
\begin{definition}
	Let $\A$ be a either a $C^*$-algebra or a $W^*$-algebra, $\alpha$ be a dynamical evolution on $\A$, $\omega$ be a state which we assume to be normal if $\A$ is a $W^*$-algebra, and $\beta\in\mathbb{R}$. Define the strip
	\begin{equation}
	\mathbb{C}\supseteq\overline{\mathcal{D}}_\beta:=\begin{cases}
	\mathbb{R}\times(0,\beta), & \beta>0\\
	\mathbb{R}\times\{0\}, & \beta = 0\\
	\mathbb{R}\times(\beta,0), & \beta<0
	\end{cases}
	\end{equation}
and $\mathcal{D}_\beta$ to be its interior. Then, $\omega$ is said to be a $(\alpha,\beta)$-KMS state if for every $a,b\in\A$ there exists a bounded continuous function $F_{a,b}:\overline{\mathcal{D}}_\beta\rightarrow\mathbb{C}$ which is analytic on $\mathcal{D}_\beta$ and such that for all $t\in\mathbb{R}$
\begin{align}
\begin{split}
F_{a,b}(t)=&\omega(a\alpha_t(b)),\\
F_{a,b}(t+i\beta)=&\omega(\alpha_t(b)a).
\end{split}
\end{align}
\end{definition}
We identify the $(\alpha,\beta)$-KMS states as those in thermal equilibrium at inverse temperature $\beta$. It is surprising that this states are intimately related with Tomita-Takesaki theory\cite{Takesaki1970}. This theory considers a $W^*$-algebra $\M$ acting on a Hilbert space $\h$ with a vector $\Omega\in\h$ which is cyclic ($\overline{\M\Omega}=\h$) and separating ($\M\rightarrow\M\Omega$) is injective). The theory provides an antilinear isometry $J$ called the modular conjugation and a positive selfadjoint $\Delta$ called the modular operator on $\h$.
\begin{theorem}[Tomita-Takesaki Theorem]
Let $\M$ be a $W^*$-algebra acting on $\h$ and $\Omega\in\h$ be a cyclic and separating vector. Then the modular operator and modular satisfies $\Delta^{-it}\M\Delta^{it}=\M$ for all $t\in\mathbb{R}$ and the modular conjugation satisfies $J\M J=\M':=\{A\in\BH|[A,B]=0\text{ for all }B\in\M\}$.
\end{theorem}
As it turns out, the GNS construction allows us to build a representation $\pi_\omega:\M\rightarrow\mathcal{B}(\mathcal{H}_\omega)$ for every state $\omega$ on $\M$. If $\omega$ is normal and faithful this representation will be faithful, have a cyclic and separating vector $\Omega_\omega\in\h_\omega$ and $\pi(\M)$ will be a $W^*$-algebra. This allows us to build a modular operator $\Delta$ and a modular conjugation $J$ on $\h_\omega$. This will in particular define a dynamical evolution on $\M$.
\begin{definition}
Let $\omega$ be a faithful normal state on a $W^*$-algebra $\M$ and $\Delta$ and $J$ be the modular operator and modular conjugation associated to $\pi_\omega(\M)$ and $\Omega_\omega$. We define the dynamical law $\alpha^\omega$ on $\M$ by 
\begin{equation}
\alpha^\omega_t(a):=\pi_\omega^{-1}(\Delta^{-it}\pi_\omega(a)\Delta^{it}).
\end{equation}
This is called the modular group.
\end{definition}
The physical importance of this construction lies precisely in the connection between modular theory and KMS states.
\begin{theorem}
Let $\omega$ be a normal faithful state. The unique dynamical evolution $\tau$ under which $\omega$ is an $(\tau,\beta)$-KMS state is given by $\tau_t=\alpha^\omega_{t/\beta}$.
\end{theorem}
In particular, this observation led to the thermal time hypothesis.

The final main ingredient for this thesis will be a theorem by Connes\cite{Connes1973}
\begin{theorem}
	Let $\omega$ and $\mu$ be faithful normal states on a $W^*$-algebra $\M$. Then there exists a canonical 1-cocycle
	\begin{align}
	\begin{split}
		u:\mathbb{R}&\rightarrow\M\\
		t&\mapsto u_t,
	\end{split}
	\end{align}	 
	known as the Connes cocycle, such that for all $t,t'\in\mathbb{R}$ we have that $u_t$ is unitary, $u_{t+t'}=u_t\alpha^\omega_t(u_t')$, and for all $a\in\M$ 
	\begin{equation}
		\alpha^\mu_t(a)=u_t\alpha_t^\omega(a)u_t^*.
	\end{equation}
\end{theorem}
This theorem is to be understood as a generalization of the Radon-Nikod\'ym theorem. Indeed, in the commutative case we have $i(\dv*{u_t}{t})_{t=0}=\dv*{\mu}{\omega}$, identifying $\mu$ and $\omega$ with their corresponding probability measures according to the Riesz representation theorem. Moreover, in the case of statistical mechanics, this corresponds to the relative Hamiltonian of Araki\cite{Araki1973}. In particular, it shows that all modular groups are related to one another through an inner automorphism of the algebra. This leads us to the central result that every $W^*$-algebra has a cononical dynamical evolution up to inner automorphisms
\begin{equation}
\mathbb{R}\rightarrow\Out\M.
\end{equation}
This shows that $W^*$-algebras are canonically dynamical objects.


\section{Methodology}

This investigation will be centered around the individual work of the student followed with weekly meetings with the advisor and the QFT/Mathematical Physics group. Individual work will be focused on reviewing the existent bibliography and the development of the analytic and computational tools required for the calculations. In particular, the computational requirements will be implemented through the use of Python and Mathematica. On the other hand, the group work will consist on sharing the weekly progress achieved while also offering the newly obtained knowledge for the solution of the problems other members of the group may face.

\section{Timetable}

\begin{table}[htb]
	\begin{tabular}{|c|cccccccccccccccc| }
	\hline
	Objective $\backslash$ Week & 1 & 2 & 3 & 4 & 5 & 6 & 7 & 8 & 9 & 10 & 11 & 12 & 13 & 14 & 15 & 16  \\
	\hline
	1 & X & X &   &   &   &   &   &   &   &   &   &   &   &   &   &   \\
	2 &   &   & X & X &   &   &   &   &   &   &   &   &   &   &   &   \\
	3 &   &   &   &   & X &   &   &   &   &   &   &   &   &   &   &   \\
	4 &   &   &   &   &   & X & X & X &   &   &   &   &   &   &   &   \\
	5 &   &   &   &   &   & X & X & X &   &   &   &   &   &   &   &   \\
	6 &   &   &   &   &   &   &   &   & X &
X &   &   &   &   &   &   \\
	7 &   &   &   &   &   &   &   &   &   &
& X & X &   &   &   &   \\
	8 &   &   &   &   &   &   &   &   &   &
& X & X &   &   &   &   \\
	9 &   &   &   &   &   &   &   &   &   &
&   &   & X & X & X & X \\
	10 & X & X & X & X & X & X & X & X & X &
X & X & X & X & X & X & X \\
	\hline
	\end{tabular}
\end{table}
\vspace{1mm}

\begin{enumerate}
	\item Task 1: Topological aspects of $C^*$-algebras and $W^*$-algebras.
	\item Task 2: Spectral theory of $C^*$-algebras and $W^*$-algebras.
	\item Task 3: Representation theory of $C^*$-algebras and $W^*$-algebras.
	\item Task 4: Tomita-Takesaki theory in the unbounded operator approach.
	\item Task 5: KMS states and the modular group.
	\item Task 6: Examples of the thermal hypothesis.
	\item Task 7: Canonical dynamics of $W^*$-algebras.
	\item Task 8: Classification of $W^*$-algebras.
	\item Task 9: Construction of physical examples of canonical dynamics in $W^*$-algebras.
	\item Task 10: Document.
\end{enumerate}

\section{Expected Results}

\begin{enumerate}
\item A thorough review of the minimal theory of operator algebras required for mathematically rigorous investigations in algebraic quantum field theory and infinite statistical systems.
\item A solid exposition of the scope of the thermal time hypothesis and its relevance to physics through explicit examples.
\item A collection of examples of canonical dynamical laws induced by physically relevant noncommutative algebras of observables.
\item An exampled based discussion of Connes' emergent time and its relation to physical time.
\end{enumerate}

\section*{Recommended Jury}

\subsection*{Universidad de los Andes}

\begin{enumerate}

\item Alexander Cardona Guio

Associate Professor of the Mathematics Department

\href{mailto:acardona@uniandes.edu.co}{acardona@uniandes.edu.co}

CV: \url{https://pentagono.uniandes.edu.co/~acardona/ACcurriculumvitae.pdf}

\item Nelson Javier Buitrago Aza

Postdoctoral Researcher of the Physics Department

\href{mailto:nj.buitragoa@uniandes.edu.co}{nj.buitragoa@uniandes.edu.co}

\item Monika Winklmeier

Associate Professor of the Mathematics Department

\href{mailto:mwinklme@uniandes.edu.co}{mwinklme@uniandes.edu.co}

CV: \url{https://matematicas.uniandes.edu.co/~mwinklme/index.php} 

\end{enumerate}

\subsection*{Other}

\begin{enumerate}

\item Aiyalam Parameswaran Balachandran

Professor Emeritus at Syracuse University

\href{mailto:balachandran38@gmail.com}{balachandran38@gmail.com}

\item Jorge Andrés Plazas Vargas

Assistant Professor at Universidad Javeriana

\href{mailto:jorge.plazas@javeriana.edu.co}
{jorge.plazas@javeriana.edu.co}

CV: \url{http://scienti.colciencias.gov.co:8081/cvlac/visualizador/generarCurriculoCv.do?cod_rh=0001544588}

\item Leonardo Arturo Cano García 

Professor at Univesidad Nacional

\href{mailto:lcanog@unal.edu.co}{lcanog@unal.edu.co}

CV: \url{http://scienti.colciencias.gov.co:8081/cvlac/visualizador/generarCurriculoCv.do?cod_rh=0001450991}

\end{enumerate}

\section*{Signatures}

\vspace{2cm}

Iván Mauricio Burbano Aldana \hspace{2cm} Andrés Fernando Reyes Lega 

\bibliography{../Mendeley/library}
\bibliographystyle{unsrturl}

\end{document}
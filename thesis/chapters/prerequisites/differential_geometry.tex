\subsection{Differential Structure of Finite Dimensional Vector Spaces}

We will at first be interested in studying the usual concepts of differential calculus on finite dimensional vector spaces over a field $\mathbb{F}$ which is either $\mathbb{R}$ or $\mathbb{C}$. As we will see, the theory is completely equivalent to that of $\mathbb{R}^n$. Although this might at first seem to make the theory not worth studying, this is not the case. In fact, the study of differential structures on vector spaces makes the theory of multivariable differential calculus more transparent and shows that the ``coordinate driven'' example of $\mathbb{R}^n$ is not special amongst vector spaces. Only after this exercise is performed does the theory of multivariable calculus deserves the, usually misplaced name, \textit{vector calculus}.

We begin our study of finite dimensional vector spaces inquiring about the topologies they can equipped with. Given that vector spaces have a linear structure, we want our topologies to respect it. Consider the following definition.

\begin{definition}
	Let $V$ be a vector space and $S\subseteq V$. We say $x$ is an internal point of $S$ if for every $y\in V$ there exists an $R\in\mathbb{R}^+$ such that for all $r\in(0,R)$ we have that $x+ry\in S$. 
\end{definition}

This of course resembles the notion of an interior point in topology. However, internal points are fully defined in terms of the linear structure. If we want to have a relationship between the notions of internal and interior points we have to restrict the topologies we consider. Therefore, in order to relate the algebraic and topologically induced geometries on a vector space we need to consider linear topologies.
	
\begin{definition}
	A linear topology on a vector space $V$ is one in which both addition
	\begin{align}
	\begin{split}
		+:V\times V&\rightarrow V\\
		(v,u)&\mapsto v+u
	\end{split}
	\end{align}
and multiplication
	\begin{align}
	\begin{split}
		\mathbb{F}\times V&\rightarrow V\\
		(k,v)&\mapsto kv
	\end{split}
	\end{align}
are continuous. A vector space equipped with a linear topology is called a topological vector space.
\end{definition}

Topological vector spaces are at the heart of functional analysis. However, our main focus at the moment is the theory of finite dimensional vector spaces. In this case, we will see that the linear geometry of a vector space completely determines its topological structure. 

The usual definition of derivative in multivariable calculus requires the use of the standard norm on $\mathbb{R}^n$. Our first task will be to show that, as far as differential calculus is concerned, there is nothing special with this norm. We begin by defining an equivalence of norms through the same method equivalence of metrics is defined.

\begin{definition}\label{eq:equivalence_norm}
	Let $V$ be a vector space. A norm $\|\cdot\|$ on $V$ is said to be equivalent to a norm $\|\cdot\|'$ on $V$ if there exists constants $a,b\in\mathbb{R}^+$ such that
	\begin{equation}
		a\|v\|\leq\|v\|'\leq b\|v\|
	\end{equation}
for all $v\in V$.
\end{definition}

This is in fact an equivalence relation. 

\begin{theorem}
	The equivalence defined on definition \ref{eq:equivalence_norm} is an equivalence relation on the set of norms of $V$.
\end{theorem}

\begin{proof}
	Let $\|\cdot\|$ be a norm on $V$. Then for all $v\in V$
	\begin{equation}
		1\|v\|\leq\|v\|\leq1\|v\|
	\end{equation}
showing that equivalence of norms is reflexive. If $\|\cdot\|'$ is a norm on $V$ to which $\|\cdot\|$ is equivalent there exists $a,b\in\mathbb{R}^+$ such that for all $v\in V$
	\begin{equation}
		a\|v\|\leq\|v\|'\leq b\|v\|.
	\end{equation}
Therefore, for all $v\in V$
	\begin{equation}
		\frac{1}{b}\|v\|'\leq\|v\|\leq\frac{1}{a}\|v\|'
	\end{equation}
showing that $\|\cdot\|'$ is equivalent to $\|\cdot\|$. Thus, the equivalence of norms is reflexive. Finally assume $\|\cdot\|''$ is a norm on $V$ equivalent to $\|\cdot\|'$. Then there exists $c,d\in\mathbb{R}^+$ such that for all $v\in V$
	\begin{equation}
		c\|v\|'\leq\|v\|''\leq d\|v\|'.
	\end{equation}
Therefore for all $v\in V$
	\begin{equation}
		ca\|v\|\leq c\|v\|'\leq\|v\|''\leq d\|v\|'\leq db\|v\|.
	\end{equation}
We conclude that $\|\cdot\|''$ is equivalent to $\|\cdot\|$ and the equivalence of norms is transitive.
\end{proof}

Another ingredient we are gonna need is the fact that every norm induces a metric and thus a topology.

\begin{theorem}
	Let $V$ be a normed vector space. Then the function 
	\begin{align}\label{eq:norm_metric}
	\begin{split}
		d:V\times V&\rightarrow V\\
		(x,y)&\mapsto d(x,y):=\|x-y\|
	\end{split}
	\end{align}
is a metric on $V$. Moreover, the topology on $V$ induced by this metric makes $V$ a topological vector space.
\end{theorem}

\begin{proof}
	Given that $\|\cdot\|:V\rightarrow\mathbb{R}^+_0$ we have that for all $x,y\in V$ 
	\begin{equation}
		d(x,y)=\|x-y\|\geq 0.
	\end{equation}
Moreover, $\|x\|=0$ if and only if $x=0$ for all $x\in V$. This implies that $\|x-y\|=d(x,y)=0$ if and only if $x-y=0$, or in other words, $x=y$ for all $x,y\in V$. We also have from the definition of a norm that $\|kx\|=|k|\|x\|$ for all $k\in\mathbb{F}$ and $x\in V$. Thus, for all $x,y\in V$
	\begin{equation}
		d(x,y)=\|x-y\|=\|-(y-x)\|=|-1|\|y-x\|=\|y-x\|=d(y,x).
	\end{equation}
Finally, we have the triangle inequality $\|x+y\|\leq\|x\|+\|y\|$ for all $x,y\in V$. Therefore, for all $x,y,z\in V$
	\begin{align}
	\begin{split}
		d(x,z)=&\|x-z\|=\|(x-y)+(y-z)\|\\
		\leq&\|x-y\|+\|y-z\|=d(x,y)+d(y,z).
	\end{split}
	\end{align}
We conclude that $d$ is a metric on $V$.

	We now consider the topology induced by $d$ on $V$. Let $a,b\in V$ and $\epsilon\in\mathbb{R}^+$. For all $x,y\in V$ we have that if $\|x-a\|<\epsilon/2$ and $\|y-b\|<\epsilon/2$, then the triangle inequality guarantees
	\begin{align}
		\|(x+y)-(a+b)\|=&\|(x-a)+(y-b)\|\leq\|x-a\|+\|y-b\|\\
		<&\epsilon/2+\epsilon/2=\epsilon.
	\end{align}
Thus, the sum is continuous. Now let $a\in V$, $k\in\mathbb{F}$ and $\epsilon\in\mathbb{R}^+$. For all $l\in\mathbb{F}$ and $x\in V$ if 
	\begin{align}
	\begin{split}
		\|x-a\|<&\qty(\sqrt{1+\frac{\epsilon}{|k|\|a\|}}-1)\|a\|,\\
		|l-k|<&\qty(\sqrt{1+\frac{\epsilon}{|k|\|a\|}}-1)|k|
	\end{split}
	\end{align}
then the triangle inequality once again guarantees
	\begin{align}
	\begin{split}
		\|lx-ka\|=&\|(l-k)(x-a)+kx+la-ka-ka\|\\
		=&\|(l-k)(x-a)+k(x-a)-(l-k)a\|\\
		\leq&|l-k|\|x-a\|+|k|\|x-a\|+|l-k|\|a\|\\
		<&\qty(\sqrt{1+\frac{\epsilon}{|k|\|a\|}}-1)^2\|a\||k|+2\qty(\sqrt{1+\frac{\epsilon}{|k|\|a\|}}-1)\|a\||k|\\
		=&\qty(\sqrt{1+\frac{\epsilon}{|k|\|a\|}}+1)\qty(\sqrt{1+\frac{\epsilon}{|k|\|a\|}}-1)\|a\||k|\\
		=&\qty(1+\frac{\epsilon}{|k|\|a\|}-1)\|a\||k|=\epsilon.
	\end{split}
	\end{align}
This shows that scalar multiplication is also continuous and $V$ is a topological vector space when equipped with the topology induced by $d$. 
\end{proof}

\begin{definition}
	Let $V$ be a normed space. Then the metric $d$ defined by \eqref{eq:norm_metric} is called the metric induced by the norm of $V$.
\end{definition}

In particular, we now see that equivalence of norms precisely corresponds to equivalence of metrics. This means that if two norms are equivalent, they induce equivalent metrics, and in particular, these metric induce the same topologies. Now we are ready to show that all norms are equivalent on finite dimensional vector spaces.

\begin{theorem}
	Let $V$ be a finite dimensional vector space. Then all norms on $V$ are equivalent.
\end{theorem}

\begin{proof}
	Given that equivalence of norms is an equivalence relation, we can fix a norm on $V$ and show that every other norm is equivalent to it. To construct this norm choose a basis $\{v_1,\dots,v_n\}$ of $V$. Define
	\begin{align}
	\begin{split}
		\|\cdot\|:V&\rightarrow\mathbb{R}^+_0\\
		v&\mapsto\sum_{i=1}^n|k_i|
	\end{split}
	\end{align}
where $k_1,\dots,k_n\in\mathbb{F}$, the field over which $V$ is defined, are the unique scalars such that
	\begin{equation}
		v=\sum_{i=1}^n k_iv_i.
	\end{equation}	 
From the fact that the codomain of $|\cdot|$ is $\mathbb{R}^+_0$ we have that the codomain of $\|\cdot\|$ is $\mathbb{R}^+_0$ as well. Since $0=\sum_{i=1}^n0v_i$ we have $\|0\|=0$. Now let $v=\sum_{i=1}^nk_iv_i\in V$. If $0=\|v\|=\sum_{i=1}^n |k_i|$, we have that $k_1=\dots=k_n=0$ since the elements of the sum are non-negative. Thus $v =0$. If $k\in\mathbb{F}$ then
	\begin{equation}
		\|kv\|=\|\sum_{i=1}^nkk_iv_i\|=\sum_{i=1}^n|kk_i|=|k|\sum_{i=1}^n|k_i|=|k|\|v\|.
	\end{equation}
Finally, of $u=\sum_{i=1}^n l_iv_i\in V$ then
	\begin{align}
	\begin{split}
		\|v+u\|=\left\lVert\sum_{i=1}^n(k_i+l_i)v_i\right\rVert=\sum_{i=1}^n|k_i+l_i|\leq\sum_{i=1}^n(|k_i|+|l_i|)=\|v\|+\|u\|.
	\end{split}
	\end{align}
We have thus show that $\|\cdot\|$ is a norm on $V$.

	Now let $\|\cdot\|'$ be any other norm on $V$. Let $M:=\max\{\|v_i\|'|i\in\{1,\dots,n\}\}$. Then for any vector $v=\sum_{i=1}^nk_iv_i\in V$ we have
	\begin{align}
	\begin{split}
		\|v\|'=\left\lVert\sum_{i=1}^nk_iv_i\right\rVert'\leq\sum_{i=1}^n|k_i|\|v_i\|'\leq M\sum_{i=1}^n|k_i|=M\|v\|.
	\end{split}
	\end{align}
In particular, for any $u,v\in V$ we have
	\begin{equation}
		|\|u\|'-\|v\|'|\leq\|u-v\|'\leq M\|u-v\|.
	\end{equation}
Let $\epsilon\in\mathbb{R}^+$ and $v\in V$. Then for every $u\in V$ such that $\|u-v\|<\epsilon/M$ we have
	\begin{equation}
		|\|u\|-\|v\||\leq M\|u-v\|<M\epsilon/M=\epsilon.
	\end{equation} 
Therefore, $\|\cdot\|'$ is continuous with respect to the topology on $V$ induced by the metric induced by $\|\cdot\|$. Consider now the unit sphere $S:=\{v\in V|\|v\|=1\}=\|\cdot\|^{-1}(\{1\})$ centered at $0$ according to the metric induced by $\|\cdot\|$. It is clearly bounded in the metric induced by $\|\cdot\|$ and, given that it is the preimage of a closed set under a continuous function, closed in the topology induced by this metric. Therefore, $S$ is compact in this topology and the continuity of $\|\cdot\|'$ implies $a:=\inf\|S\|',b:=\sup\|S\|'\in\mathbb{R}$. Moreover, since $a$ and $b$ are both limit points of $\|S\|'\subseteq\mathbb{R}^+_0$, the fact that $\mathbb{R}^+_0$ is closed implies $a,b\in\mathbb{R}^+_0$. Finally, for all $v\in V$
\begin{equation}
a\|v\|\leq\left\lVert\frac{1}{\|v\|}v\right\rvert'\|v\|=\|v\|'=\left\lvert\frac{1}{\|v\|}v\right\rvert'\|v\|\leq b\|v\|.
\end{equation}
We conclude $\|\cdot\|$ and $\|\cdot\|'$ are equivalent.
\end{proof}

To the usual definition of derivative in multivariable calculus to vector spaces we are gonna have to refer to a norm structure. As it turns out, the derivative is independent of the norm chosen.

\begin{theorem}
Let $V,W$ be finite dimensional vector spaces, $v\in S\subseteq V$, and $f:S\rightarrow W$. Suppose that $T:V\rightarrow W$ is a linear transformation such that 
\begin{equation}
	0=\lim_{h\rightarrow 0}\frac{f(v+h)-f(v)-Th}{\|h\|}
\end{equation}
where $\|\cdot\|$ is a norm on $V$. Then for any other norm $\|\cdot\|'$ on $V$
\begin{equation}
	0=\lim_{h\rightarrow 0}\frac{f(v+h)-f(v)-Th}{\|h\|'}.
\end{equation}
\end{theorem}

\begin{proof}
Equip $W$ with any norm $\|\cdot\|_V$. Since $\|\cdot\|'$ is equivalent to $\|\cdot\|$ there exists $a\in\mathbb{R}^+$ such that
\begin{equation}
	a\|h\|\leq\|h\|'
\end{equation}
for all $h\in V$. Therefore, for all $h\in V$
\begin{equation}
	\left\lVert\frac{f(v+h)-f(v)-Th}{\|h\|'}\right\rVert_V\leq\frac{1}{a}\left\lVert\frac{f(v+h)-f(v)-Th}{\|h\|}\right\rVert_V.
\end{equation}
Let $\epsilon,\delta\in\mathbb{R}^+$ such that for every $h\in V$ with $\|h\|<\delta$ we have
\begin{equation}
	\left\lVert\frac{f(v+h)-f(v)-Th}{\|h\|}\right\rVert<a\epsilon.
\end{equation}
Therefore, for every $h\in V$ with $\|h\|<\delta$
\begin{equation}
	\left\lVert\frac{f(v+h)-f(v)-Th}{\|h\|'}\right\rVert_V\leq\frac{1}{a}\left\lVert\frac{f(v+h)-f(v)-Th}{\|h\|}\right\rVert_V<\frac{1}{a}a\epsilon=\epsilon.
\end{equation}
We conclude that 
\begin{equation}
	0=\lim_{h\rightarrow 0}\frac{f(v+h)-f(v)-Th}{\|h\|'}.
\end{equation}
\end{proof}

\subsection{Manifolds and their Philosophy}

In differential geometry we are interested in the study of spaces that locally resemble finite dimensional vector spaces. Of course, the implementation of locality is done through a topological structure. 

\begin{definition}

	A locally euclidean space is a topological space $X$ for which every point $p\in M$ has an open neighborhood homeomorphic to an open subset of a finite dimensional vector space $V$.

\end{definition}



 
\documentclass{article}

\usepackage{physics}
\usepackage{amssymb}

\title{Quantum Entropic Ambiguities and Tomita Takesaki Theory}
\author{Souad Maria Tabban\\Co-authors: Ph.D. Andr\'es Reyes Lega \& Iv\'an Burbano Aldana
}

\begin{document}

\maketitle

Given an algebra of observables $\mathcal{A}$ and a state $\omega$, a density matrix $\rho_\omega$ acting on  $\mathcal H_\omega$ can be  
obtained through the GNS construction $(\mathcal{H}_\omega,\pi_\omega)$ that gives rise to the same expectation values as $\omega$ for elements $a \in
\mathcal{A}$, that is, $\omega(a)\equiv\tr_{\mathcal{H}_\omega}\left(\rho_\omega\pi_\omega(a)\right)$. We can assign an entropy to the state $\omega$
by computing the von Neumann entropy of the density matrix $S(\rho_\omega)=-\tr_{\mathcal{H}_\omega}(\rho_\omega\log\rho_\omega)$. However, this density matrix is not unique, which leads to an ambiguity in the previous entropy. This occurs whenever the irreducible
components of the representation $\pi_\omega$ appear in  $\mathcal{H}_\omega$ with multiplicities (Balachandran et. al. 2013). In such cases this
ambiguity is a consequence of a gauge symmetry. We will interpret these entropy ambiguities via the Tomita-Takesaki modular theory through the action of the commutant algebra $\mathcal{A}'$ (which is a gauge
algebra given that it does not change the state) of the algebra $\mathcal{A}$ of bounded operators on a finite dimensional Hilbert space $\mathcal{H}$.
We will also obtain the ambiguity in the analogue problem of a
bipartite system through a cyclic vector induced isomorphism $\mathcal{H}_\omega\rightarrow\mathcal{H}\otimes\mathcal{H} $.  Here both the state on the full system and its restriction on the gauge system will show the phenomena. We will extend this result to the study of the ethylene molecule and explore its possible relation to anomalies.

\end{document}
\documentclass{letter}

\usepackage[utf8]{inputenc}
\usepackage{hyperref}

\signature{Iván Mauricio Burbano Aldana}
\address{Cra. 19 \# 63-27, Bogotá, Colombia}

\begin{document}

\begin{letter}{2019 Perimeter Scholars International \\ Perimeter Institute for Theoretical Physics}

\opening{Dear Admissions Committee:}

I first watched professor Bender's incredible lectures on mathematical physics during my first year as an undergraduate. Ever since then, I've known that perimeter would be the right place for me. It is now my turn to show you why I am the right student for Perimeter.

The research I've done up till now lies on the intersection between mathematics and physics. I've particularly focused on the algebraic approach to quantum physics. During my undergraduate degree, my thesis was based on the relationship between KMS states and Tomita-Takesaki theory. That's where I first learned of the thermal time hypothesis. The beauty behind it led me to the research I am doing at the moment. I am currently working on a paper with A. P. Balachandran, A. F. Reyes-Lega and S. Tabban where we explore the appearence of gauge symmetries when calculating the entropy of algebraic states. We further explain these in terms of quantum operations. This work is deeply rooted within the Perimeter community, as one of the starting points was the ambiguity pointed out by Prof. R. Sorkin on our entropy calculations. On the other hand, I believe these ambiguities may be connected to Connes' proposal of quantum time as a one parameter group of {\bf outer} automorphisms. This is the result of the application of noncommutative geometry to physics, a research field to which Prof. M. Marcolli has been instrumental.

I applied last year to the program. Although my application was received as exceptional, Perimeter was unable to offer me a position. As a result of my participation in the school \href{http://journeys.ictp-saifr.org/}{Journeys into Theoretical Physics}, I was awarded a fellowship for a joint masters program between ICTP-SAIFR/IFT-UNESP and Perimeter. I am truly grateful for this second opportunity and I promise to make the best out of it. I expect to be able to learn as much as I can from the incredible scientists mentioned above. Moreover, I believe this is a great opportunity to forge scientific relations with them which may lead to new research.

\closing{Sincerely,}

\end{letter}

\end{document}